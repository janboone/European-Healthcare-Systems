% Created 2021-04-16 vr 15:13
% Intended LaTeX compiler: pdflatex
\documentclass{article}
  \documentclass[12pt,english,a4paper]{article}
\usepackage[utf8]{inputenc}
\usepackage{sectsty}
\sectionfont{\normalfont\scshape}
\subsectionfont{\normalfont\itshape}
\usepackage[round,authoryear]{natbib}
\usepackage{amsmath}
\newtheorem{theorem}{Theorem}
\newtheorem{assumption}{Assumption}
\newtheorem{acknowledgement}{Acknowledgement}
\newtheorem{algorithm}{Algorithm}
\newtheorem{axiom}{Axiom}
\newtheorem{case}{Case}
\newtheorem{claim}{Claim}
\newtheorem{conclusion}{Conclusion}
\newtheorem{condition}{Condition}
\newtheorem{conjecture}{Conjecture}
\newtheorem{corollary}{Corollary}
\newtheorem{criterion}{Criterion}
\newtheorem{definition}{Definition}
\newtheorem{example}{Example}
\newtheorem{exercise}{Exercise}
\newtheorem{lemma}{Lemma}
\newtheorem{notation}{Notation}
\newtheorem{observation}{Observation}
\newtheorem{problem}{Problem}
\newtheorem{proposition}{Proposition}
\newtheorem{remark}{Remark}
\newtheorem{result}{Result}
\newtheorem{summary}{Summary}
\newtheorem{Hypothesis}{Hypothesis}
\newcommand{\qed}{\hspace*{\fill} {\em Q.E.D.}}
\usepackage{pdftexcmds}
\usepackage{minted}
\usepackage{textcomp}
\usepackage{hyperref}
\usepackage{graphicx}
\linespread{1.3}
\usepackage[right=3cm, left=3cm,top=3cm,bottom=3cm]{geometry}
\newcommand{\hf}{{\textstyle{1\over 2}}}
\newcommand{\kw}{{\textstyle{1\over 4}}}
\newcommand{\hs}{{\textstyle{1\over 6}}}
\newcommand{\hr}{{\textstyle{1\over 3}}}
\author{Jan Boone\inst{*}}
\date{\today}
\title{Comparing European healthcare systems}
\hypersetup{
 pdfauthor={Jan Boone\inst{*}},
 pdftitle={Comparing European healthcare systems},
 pdfkeywords={},
 pdfsubject={},
 pdfcreator={Emacs 27.2 (Org mode 9.4.4)}, 
 pdflang={English}}
\begin{document}

\maketitle
\begin{PREFACE}
\maketitle
\end{PREFACE}



 \begin{abstract}

We use recent data from Eurostat and OECD to compare the performance of European healthcare systems on policy variables related to efficiency, quality and competition. We find that --conditional on healthcare expenditure-- the following efficiency measures tend to improve performance in terms of reduced mortality: cost effectiveness analysis to determine which treatments are covered by insurance, increasing the ratio of nurses over doctors and a well developed primary care sector. Introducing a regulator specifically for healthcare quality and making provider quality reports public also improve outcomes. In terms of competition, allowing patients to choose their provider leads to lower mortality. But in terms of insurer competition, single payer systems tend to do better than multiple insurers, especially in countries where inpatient care is predominantly delivered by public hospitals.

 \end{abstract}


\textbf{JEL codes:} I11, I13, I18

\textbf{Keywords:} healthcare systems, insurer competition, single payer, hospital choice


\vspace*{\fill}
\institute{\inst{*} Department of Economics, Tilec, Tilburg University and CEPR, London, United Kingdom}

\newpage

\section{Introduction}
\label{sec:org520754a}

COVID-19 has shown that the damage caused by a health crisis is strongly affected by the performance of a country's healthcare system \citep{OECD_2020}. This paper compares the performance of a number of health system features related to competition, efficiency and quality using data on European countries from before COVID-19. Based on their prominence in \cite{countryprofileReport}, its accompanying country reports and \citep{OECD_2020}, we analyze the effects of policy choices like single payer vs. competing health insurers, care delivery via private or public providers, introducing a dedicated healthcare regulator, allowing patients to choose their own provider, the prominence of primary care in the system and the employment of nurses vs. doctors. We use a Bayesian model to quantify the uncertainty on these effects from comparing macro data for a set of countries. This allows us to answer a question like: how sure are we that introducing a separate regulator dedicated to healthcare improves the system's performance?

As motivated below, we use three different measures of mortality to quantify the performance of a healthcare system. We find the following effects. Conditioning on healthcare expenditure, efficiency related measures like increasing the ratio of nurses to doctors, improving primary care and using cost effectiveness analysis to allow new treatments tends to improve healthcare performance. Allowing people to choose their provider and making provider quality reports public are associated with lower mortality; but these effects are not additive. In other words, provider choice and public provider reports are partial substitutes. For inpatient care there is a clear indication that the combination of a single payer with (predominantly) public providers goes hand in hand with reduced mortality.

The motivation for this analysis is twofold. First, there are now consistent and fairly complete data available on healthcare policies and related variables for European countries together with in-depth country reports to interpret the data and the underlying institutional setting. OECD has gathered data on a number of health policies across countries and Eurostat provides data on mortality, income, income inequality etc. across countries.

Second, there are numerous studies on particular aspects of healthcare systems aimed at identifying with precision the causal effect of a policy parameter on a particular outcome. Such papers tend to use individual level data and research designs include difference-in-differences, regression discontinuity, instrumental variables \citep{econometricevaluation}. This is not the goal of the current paper. The idea here is more broad brush. Is there an effect of allowing cost effectiveness considerations into the admission of new treatments? Is there a role for (better) primary care? When does a single payer system perform better than competing insurers? Is this affected by the public/private status of providers? Because healthcare is such a big part of the economy, if such policies have an effect we expect to see this at the macro level, for instance in aggregate mortality outcomes. 

There a number of disadvantages of this approach. First, it is not very precise. To illustrate, a policy change can have a meaningful impact, but does not cause an observable change in macro numbers. Second, our approach does not prove causality which can be established in a (quasi) experimental set-up. The policy variables analyzed here are actually important components of any healthcare system,\footnote{See \url{https://www.theguardian.com/society/2021/feb/12/coronavirus-shows-what-we-are-capable-of-nhs-vaccine-triumph} arguing that the UK's primary care network was instrumental in the fast corona vaccine roll-out.} but there may not be (quasi) experimental data with which their effect can be evaluated. To illustrate, a country either has a (dedicated) healthcare regulator or it does not. It is not possible to compare individuals with and without such a regulator. Further, randomized trials help to identify individual (micro) effects but can not always identify general equilibrium (macro) effects.

Comparing healthcare systems used in different countries can help to learn something about the macro effects of these key policy choices. Advantages of using our approach with country level data are the following. First, we can evaluate overall system performance; for instance, by considering interaction effects between policies. Second, although mortality is a crude measure of system performance, we do find population wide effects of the policies. That is, we do not need to extrapolate from the sample to the population average treatment effect. This approach is feasible now because comparable data across countries are available from Eurostat and OECD. An advantage of using these data is that they are available to everyone which makes our analysis reproducible; this is not always the case with individual level data. To illustrate, if a policy maker would like to change our analysis e.g. by adding a policy variable, this is easy to do given that our code and data can be adapted.

For two reasons we use a Bayesian model to summarize our cross country comparisons. First, it allows us to quantify the remaining uncertainty of the posterior distribution. We want to be able to say things like: with 90\% probability, the introduction of a separate healthcare regulator is associated with better system performance. Second, although the OECD and Eurostat have made great strides in publishing consistently gathered data across a number of countries, this data is not perfect. That is, the data is quite recent (no decades long time series) and there are missing observations. Dropping all records that are not complete would make the data set too  small. A Bayesian model can deal with missing observations without imputing or interpolating numbers. Intuitively, the posterior draws a value for the missing variable from a distribution representing the remaining uncertainty around this variable. If we sample, say, 2000 values for the posterior, we get 2000 draws for missing values. In this way, the uncertainty surrounding the missing value is taken into account in the uncertainty of the estimated parameters.

Since the consistent --across a number of countries-- healthcare data by Eurostat and the OECD is quite recent, we are not aware of papers that try to compare the performance of different healtcare systems in this way. Of course, there are papers analyzing elements of healtcare systems which we discuss below.

We split our independent variables in two subsets. The first consists of the policy variables that we are interested in which are related to competition, quality and efficiency. The second subset consists of the "usual suspects" when controlling for other effects. Think of GDP per head, income inequality, healtcare expenditure, lifestyle variables related to body mass index, alcohol and tobacco consumption.

For the latter variables we know that identifying the causal relations between these variables and health status (measured by mortality) is complicated. Does higher income cause better health or is it the case that a more healthy population is more productive and hence produces higher GDP per capita? See, for instance, \cite{socioeconomicstatus} for an overview. Is healthcare expenditure caused by the health of the population or does it reduce mortality in the population? We will not disentangle these effects; we just control for these variables.

The policy variables we are interested in are the following. The ones related to efficiency are the use of nurses vs. physicians, cost effectiveness (CE) analysis when deciding on treatment coverage and the development of primary care to coordinate and contain healthcare expenditure. Although \cite{countryprofileReport} and the accompanying country reports suggest that it is a good idea to increase the ratio of nurses to doctors, we are not aware of country studies showing that this improves the performance of a healthcare system. Our analysis shows that for given healthcare expenditure, an increase in the ratio of nurses to physicians reduces mortality. The literature on CE analysis focuses on how this analysis can be done in practice; see, for instance, \cite{Drummond2005} and \cite{Gold96e}. But no evidence is provided that doing CE analysis actually improves the performance of the healthcare system. This is what we look at in this paper. The picture that emerges from papers on primary care is mixed. There is evidence that access to primary care improves health outcomes; see \cite{starfield2005} and references therein. But \cite{AAKVIK20061139} do not find a clear relation between the number of general practitioners and mortality. Although gatekeeping is associated with lower expenditure per head, it is not clear whether gatekeeping causes lower expenditure or a country's low expenditure induces it to introduce gatekeeping \citep{Forrest692}. As explained below, we use avoidable hospitalizations to measure how well developed primary care is in a country. We find that for given expenditure, better developed primary care reduces mortality.

The variables related to competition are whether a patient is free to choose her preferred provider, whether there is a single payer or a number of (competing) insurers and whether healthcare is mainly organized via public or private organizations. \cite{NBERw19800} review the literature on competition effects in healthcare markets. A number of studies find that more provider and more insurer competition lead to better outcomes. We also find that giving patients freedom to choose their hospital reduces mortality. The effects of insurer competition compared to a single payer are not so clear. A number of papers argue that single payer systems lead to better outcomes; see, for instance, \cite{BICHAY2020113454} and \cite{OberlanderSinglePayer}. Arguments for a single payer system include a better bargaining position of the payer vis-a-vis providers and the absence of risk selection incentives. But insurer competition is also supposed to give incentives for insurers to contract high quality care at low prices. For given expenditure, we find that a single payer reduces mortality. This is the case especially for inpatient care when physicians are predominantly publicly employed.

The policy variables related to quality are whether provider quality reports are made public and whether a country has a regulator dedicated to healthcare. We are not aware of studies focusing on the effect of introducing a regulator dedicated to healthcare. On the one hand, making provider quality transparent for patients seems like a good idea but it can lead to a strategic reaction by doctors and hospitals leading to a worse outcome \citep{dranoveReportCards2003}. We find at the macro level that making reports public reduces mortality but less so if patients are free to choose their providers.


\begin{table}[htbp]
\caption{\label{tab:org84e05ef}Four countries have the optimal binary choices}
\centering
\begin{tabular}{lrr}
Country & avoidable hospitalizations & nurses/doctor ratio\\
\hline
Iceland & 200.48 & 4.29466\\
Italy & 74.08 & 1.45551\\
Lithuania & 291.567 & 1.83594\\
Slovenia & 139.817 & 3.2136\\
\end{tabular}
\end{table}

Table \ref{tab:org84e05ef} shows the four countries in our data that implement the binary policy choices that our analysis suggests are optimal. They use CE analysis, allow patients to choose their providers, inpatient and primary care are predominantly publicly provided, have a separate healthcare regulator and a single payer system. The two remaining main variables are avoidable hospitalizations and the nurses/doctor ratio. The former should be low and the latter high to minimize mortality. Italy scores well on the avoidable hospitalizations, while Iceland and Slovenia feature high nurses/doctor ratios. This does not imply that these countries have the lowest mortality rates as these are also affected by income, healthcare expenditure and lifestyle choices. Further, a country may not have a single payer system, but such low avoidable hospitalizations that it has lower mortality than these four countries. The table does suggest that it is possible to implement the optimal combinations in practice.

In terms of methods, we use a Bayesian analysis as explained in \cite{mcelreath} and \cite{GelmanBook}. In terms of software we use python and the pymc3 library developed by \cite{pymc3}. The code and data for this paper can be found in the following repository: \url{https://github.com/janboone/European-Healthcare-Systems}.

The next section provides the theoretical background for our analysis of healthcare system features on mortality. Then we describe the data sources and variables that we use for this analysis and introduce the model that we estimate. We discuss the results from the estimated model and the policy implications. We finish with a robustness analysis.



\section{Theory}
\label{sec:orgc882928}
\label{sec:theory}

The goal of this paper is to shed light on some major factors that (are supposed to) improve the functioning of a healthcare system. This leads to three types of variables for our analysis. First, we need a measure of success of the system. Second, there are variables we need to control for and third there are the health policy variables that we are interested in.

There does not exist a simple measure summarizing how successful a healthcare system is. A measure one can think of is mortality. One goal of the healthcare system is to prevent people from dying prematurely. Of course, there are other healthcare goals like treating patients with respect, creating quality of life (not just a long life), caring for people that cannot be cured even if this does not lead to more life years. While acknowledging these shortcomings, we work with mortality as our measure of healthcare performance. In particular, in the main text we work with "treatable mortality" which, among other things, corrects for the age distribution in the population. Hence, we will not take age into account in our model. We come back to the definition of treatable mortality in the data section, but, to illustrate, a ninety year old who dies is not part of this statistic. Indeed, such a death is hardly a "failure" of the healthcare system. 

From reading country health reports like \cite{countryprofileUK} one can distill a number of factors that are (deemed to be) important in determining a healthcare system's success. We make the following --to some extent arbitrary-- distinction between variables that we control for and policy variables that we are interested in determining the success of the system.

We control for income per head since people with higher incomes tend to live healthier in a number of ways \citep{socioeconomicstatus}. We also control for income inequality: for given average income, higher inequality implies more people on low incomes which tend to live less healthy. In addition, we use the following direct measures on lifestyle: alcohol consumption, smoking and body mass index (bmi). Finally, we control for healthcare expenditures per head. In other words, we focus on "health bang for your buck". Given the amount of money spent in a healthcare system, how can we maximize the health gain from this expenditure.

The policy variables that we are directly interested in are the efficiency of the system as measured by the relative number of nurses vs. doctors, the quality of primary care services and cost effectiveness analysis when deciding on covering new treatments in insurance. Policy choices that directly affect quality include the presence of a healthcare regulator and whether provider quality reports are made public. Variables that affect the extent to which healthcare works as a market: whether patients have (some) choice of providers, whether primary, inpatient and outpatient care are publicly or privately organized and whether a country has a single payer system or a number of (competing) insurers.

\subsection{Healthcare system model}
\label{sec:org6ecf6c0}

In this section we propose a model that helps us think about these variables, their underlying relationships and effects on the success of the overall system. One benefit of conditioning our analysis on healthcare expenditure is that we will ignore a number of issues related to the health insurance market. Demand side cost sharing, moral hazard and adverse selection are important dimensions of a healthcare system. However, they affect health via access to insurance and healthcare expenditure. By conditioning on expenditure we "close" this causal path \citep{Causality2009}.

\subsubsection{income (inequality)}
\label{sec:org49c43ee}

As mentioned, a major determinant of health is income, here measured as GDP per head. We do not expect an additional euro per head to reduce mortality; more that the order of magnitude has an effect. Hence, we start by considering the relation between mortality and GDP per head on a log-log scale as shown in Figure \ref{fig:org5829a41}. We have separate mortality data for men and women and the figure shows that women tend to have lower mortality than men. Further, this difference seems to be bigger the lower GDP per head is.

\begin{figure}[htbp]
\centering
\includegraphics[width=.9\linewidth]{./figures/gdp_mortality_log_log_scale_by_sex.png}
\caption{\label{fig:org5829a41}Relation between log treatable mortality (by gender) and log GDP per head.}
\end{figure}

The figure suggests a linear relation between log GDP per head and log (treatable) mortality across countries and years in our data set. If we believe that this relation holds both between and within countries, income inequality in a country tends to raise mortality. This can be seen as follows. Let \(M\) denote mortality and \(m=\ln(M)\) log mortality. Similarly, \(Y\) denotes GDP per head and \(y=\ln(Y)\). Then the relation between \(m\) and \(Y\) is of the form:
\begin{equation}
\label{eq:2}
m = \beta_0 + \beta_1 \ln(Y)
\end{equation}
with \(\beta_1 <0\). Hence, \(m\) is convex in \(Y\) and an increase in inequality (for given average \(Y\)) increases mortality:
\begin{equation}
\label{eq:3}
\beta_0 + \beta_1 \ln(\phi Y_1 + (1-\phi) Y_2) < \beta_0 + \beta_1 (\phi \ln(Y_1) + (1-\phi) \ln(Y_2))
\end{equation}
for two income levels \(Y_1 \neq Y_2\) and \(\phi \in \langle 0,1 \rangle\). In words, the gain in health due to a higher income is smaller than the health loss due to a lower income (for given average income). One reason for this is decreasing returns to health expenditure. First, money is spent on the most effective treatments; as income increases further, money is spent on less effective treatments as well. As explained below another effect is that income inequality tends to reduce providers' investments in quality.

Based on this observation, we measure inequality in our data set as:
\begin{equation}
\label{eq:19}
\text{inequality} = \ln\left(\sum_{i=1}^5 \phi_i Y_i \right) - \sum_{i=1}^5 \phi_i \ln(Y_i)
\end{equation}
where the fraction of people in income category \(i\) denoted \(\phi_i = 0.2\) as we have data on income quintiles. The higher this expression, the more unequal the income distribution in a country is.

\subsubsection{health}
\label{sec:orge53ea97}

The way we model people's health and mortality is as follows. People can be in either of two states: healthy (\(H\), fraction \(h\) of population) or low health status (\(L\), fraction \(l=1-h\)). The rate at which people flow from the \(H\) to the \(L\) state is denoted by \(\zeta\). This rate \(\zeta\) increases with lifestyle factors such as poverty (low average income and high income inequality), smoking, alcohol consumption and high BMI.

The rate at which people flow back from low to high health state is \(\tau\) where \(\tau\) denotes the rate at which patients are treated by a provider and cured. If they do not flow back to \(H\), there is a rate \(\delta > 0\) at which they die of treatable causes. Hence, only in the low health state can someone die in our set-up.

We define the steady state as \(h\) such that \(\zeta h = \tau (1-h)\). That is, to simplify expressions, we do not count dead people here because we think of \(\delta\) as being small --order of magnitude 200 over 100,000 in our data. Then we have \(h =\tau/(\zeta+\tau)\). Hence, treatable mortality rate (as fraction of healthy) equals \(M = \delta (1-h)/h = \delta \zeta/(\tau)\). Writing this in logs, we find that
\begin{equation}
\label{eq:1}
m = \ln(\delta) + \ln(\zeta) - \ln(\tau)
\end{equation}
where \(\ln(\delta)\) is a constant, \(\zeta\) is affected by lifestyle (smoking, diet and alcohol consumption), income and income inequality, treatment success \(\tau\) by expenditure \(E\) on health and efficiency measures like primary care, nurses to doctors ratio, quality policies etc. In other words, we make the choice here to say that the probability of death conditional on having low health (\(\delta\)) is the same across countries. The difference is that more successful countries have people in the low health state for a shorter period. This is a matter of presentation; it is fine to interpret some of the effects below as affecting \(\delta\) as well.

To determine the optimal choice of \(\zeta\), we view the agent as solving the following Bellman equations:
\begin{align}
\label{eq:4}
\rho V_h &= \max_{\zeta \geq 0} u_h(Y,\zeta) - \zeta \Delta V \\
\label{eq:4b}
\rho V_l &= u_l(Y) + \tau^{e} \Delta V
\end{align}
here \(\Delta V = V_h - V_l\) is the drop in expected discounted utility from high to low health state transition, \(\zeta>0\) denotes the (lack of) effort to stay healthy in the high health state and \(\tau^e\) the expected treatment success where the expectation is taken over different treatments, providers etc. Direct utility \(u_h,u_l\) is a function of income (consumption) in each health state and of effort in the high state, where utility in the high state is higher than in the low state, \(u_h(Y,\zeta) > u_l(Y)\). In the low state there is a rate \(\delta > 0\) at which an agent dies but --as mentioned-- this probability is close to zero and we ignore it in the equations here.  

In principle, demand for treatment in the \(L\) state is affected by the price of treatment in terms of out-of-pocket payments. However, in our empirical analysis below, we control for health expenditure. Hence, equation \eqref{eq:4b} does not explicitly model the demand for treatment. The probability that a patient is cured is captured by \(\tau^e\) which depends, among other things, on healthcare expenditure without distinguishing between demand side rationing (e.g. through waiting lists and out-of-pocket payments) and supply side rationing (e.g. through budgets).

Low \(\zeta\) is achieved through a healthy lifestyle; it implies a small transition rate to the low health state. However, it comes at a utility loss: \(\partial u_h/ \partial \zeta > 0\). One can think here of the disutility of changing diet, the fact that healthy food options tend to be expensive, the cost of a gym membership, (some) people's dislike of exercise. We further assume that a healthy lifestyle is a normal good: \(\partial^2 u_h/ \partial Y \partial \zeta  \leq 0\). To ensure concavity of the optimization problem we assume \(\partial^2 u_h / \partial \zeta^2 <0\). Then the first order condition for \(\zeta\) can be written as
\begin{equation}
\label{eq:5}
\frac{\partial u_h}{\partial \zeta} = \Delta V
\end{equation}
The higher the loss of utility due to low health, the more people tend to invest to prevent this (by having low \(\zeta\)). The effect of \(Y\) on the optimal \(\zeta\) then equals
\begin{equation}
\label{eq:6}
\left(- \frac{\partial^2 u_h}{\partial \zeta^2}\right) \frac{d\zeta}{dY} = \frac{\partial^2 u_h}{\partial \zeta \partial Y} - \frac{\partial \Delta V}{\partial Y} 
\end{equation}
A sufficient condition for \(d\zeta/dY \leq 0\) is that the loss from falling ill increases with income (\(\partial \Delta v/\partial Y \geq 0\); we derive this inequality below). In words, people with a higher income, choose lower \(\zeta\), that is a more healthy lifestyle. Further, government spending on prevention would tend to reduce \(\zeta\) as well.

In order to derive \(\partial \Delta V/ \partial Y\), we differentiate equations \eqref{eq:4} and \eqref{eq:4b} with respect to \(Y\) and use an envelop argument:
\begin{align}
\label{eq:25}
\rho \frac{dV_h}{dY} &= \frac{\partial u_h}{\partial Y} - \zeta \frac{d \Delta V}{d Y} \\
\label{eq:25b}
\rho \frac{dV_l}{dY} &= \frac{\partial u_l}{\partial Y} + \tau^e \frac{d \Delta V}{d Y}
\end{align}
Subtracting these two equations, we find
\begin{equation}
\label{eq:26}
\frac{d \Delta V}{dY} = \frac{\frac{\partial u_h}{\partial Y} - \frac{\partial u_l}{\partial Y}}{\rho + \zeta + \tau^e} > 0
\end{equation}
if we assume that \(\partial u_h/\partial Y > \partial u_l /\partial Y\); that is, the marginal utility of income (consumption) is higher in the healthy state than in the low health state. This seems a reasonable assumption as lack of health can reduce the set of consumption possibilities; e.g. it may be harder or not possible at all to travel, go skiing etc. when one is not fully fit. Hence the marginal utility of income is expected to be lower in the low health state and hence \(\Delta V\) increases with an individual's income. Note that the analysis below is done conditional on healthcare expenditure. Hence, the effect that high income in the low health state is useful to finance very expensive treatments disappears due to this conditioning.

\subsubsection{competition}
\label{sec:org843adb1}

To see how the probability of being cured, \(\tau\) varies with healthcare expenditure, the use of nurses, primary care etc., we model hospital competition in the following way. Consider two hospitals --denoted 1,2. To simplify the exposition, we assume that the hospitals are symmetric, meaning: (i) if they offer the same quality, patients are distributed 50:50 among the hospitals and (ii) if patients are not free to choose hospitals, they are also distributed 50:50. Think of two hospitals in two regions with similar populations in the two regions. If there is no choice, patients can only visit the hospital in the region where they live and the assumption is that the expected number of patients and their ailments is the same across regions. We denote the probability that patients are aware of quality differences between hospitals and can act upon this knowledge by \(\theta \in [0,1]\). We denote the probability of treatment success (i.e. cure and move back to state \(H\)) for hospital \(i\)  by \(\tau_i\). Then provider 1's market share is given by
\begin{equation}
\label{eq:8}
x_1 = \textstyle{1\over 2}  + \theta \Delta V (\tau_1 - \tau_2)
\end{equation}
and \(x_2 = 1- x_1\). In words, once a patient is aware of the quality difference between hospitals and is allowed to act upon this difference, the probability that she switches to provider 1 increases in the utility difference between the hospitals. Given equation \eqref{eq:4b}, the utility of visiting hospital \(i\) is given by \(\tau_i \Delta V\) and hence the utility difference is given by: \(\Delta V (\tau_1-\tau_2)\). Strictly speaking \(x_1\) is between 0 and 1 and when we do marginal comparative static analysis below, we assume this to be the case.

Instead of being treated by a specialist in a hospital, some patients can be cured either by a nurse or in primary care. We denote the fraction of patients treated by the former by \(\psi_n\) and by the latter \(\psi_{pc}\). Clearly not all patients can be cured by a nurse or primary care. For the countries in our data set there is no reason to assume that either of these alternatives has been used to such an extent that this actually increases mortality. However, there is ample evidence that these alternatives are not used to the optimal extent in many countries (see the country reports accompanying \cite{countryprofileReport}). We will not model the exact utility of treatment by nurses or primary care compared to specialists in a hospital. Although modeling this would allow us to determine the optimal mix of specialist care, nurses and primary care, we would not be able to identify such an optimal mix in the data. Hence, we assume that over the relevant range a small increase in \(\psi_n\) and \(\psi_{pc}\) has no direct effect on mortality.

We do assume that the variable cost of treatment by a specialist \(c_s\) exceeds the costs of nurses, \(c_n\) and primary care, \(c_{pc}\). Let \(R\) denote the revenue a specialist receives when treating a patient. Given that we condition on expenditure in the empirical analysis below and we assume (for simplicity) that the total budget is spent on these three alternatives for treatment:
\begin{equation}
\label{eq:9}
\frac{\zeta}{\zeta + \tau^e}(\psi_n c_n + \psi_{pc} c_{pc} + (1-\psi_n-\psi_{pc}) R) = H
\end{equation}
We endogenize the fee-for-service received by a specialist and assume that nurses and general practitioners receive a fixed remuneration equal to their cost. We then write the fee-for-service as
\begin{equation}
\label{eq:10}
R = \frac{H \frac{\zeta + \tau^{e}}{\zeta}-\psi_n c_n - \psi_{pc} c_{pc}}{1-\psi_n -\psi_{pc}} > c_s
\end{equation}
The inequality follows from the assumption that the fee-for-service covers the cost of specialist care. If this would not be the case, the system would collapse: hospitals would be loss making and close down. Taking the derivative with respect to \(\psi_i\) in equation \eqref{eq:9} it follows that
\begin{equation}
\label{eq:11}
\frac{dR}{d\psi_i} = \frac{R-c_i}{1-\psi_n - \psi_{pc}} > 0
\end{equation}
for \(i=n,pc\) because \(R>c_s>c_n,c_{pc}\).

Moreover, if the fraction of healthy people is higher (low \(\zeta\) or high \(\tau\)), for given expenditure \(H\), more money can be spent on a patient in hospital.

\subsubsection{treatment quality}
\label{sec:org3976ac8}

We write the objective function of hospital \(i \neq j\) as
\begin{equation}
\label{eq:12}
\max_{\tau_i \geq \underline \tau} \frac{\zeta}{\zeta + \tau^e}(1-\psi_n-\psi_{pc})\left( \textstyle{1\over 2}  + \theta \Delta V (\tau_i - \tau_j) \right)(R-c_s - \gamma \tau_i)
\end{equation}
where \(\underline \tau \geq 0\) denotes some minimum quality standard imposed by the government, \((1-\psi_n - \psi_{pc})\) denotes the probability that the patient is treated in hospital and hospitals can invest effort at cost \(\gamma \tau_i\) to increase the probability to cure a patient, \(\tau_i\). It is routine to verify that the first order condition for \(\tau_i\) can be written as
\begin{equation}
\label{eq:13}
\tau_i = \max \left\{ \underline \tau, \frac{R-c_s}{\gamma} - \frac{1}{2 \theta \Delta V} \right\}
\end{equation}

\subsubsection{expenditure}
\label{sec:org44129f5}

Increasing the healthcare budget increases the probability of being cured:
\begin{equation}
\label{eq:16}
\frac{d\tau_i}{dH} = \frac{1}{\gamma} \frac{dR}{dH} > 0
\end{equation}
By making more money available for healthcare, people are more likely to be cured and flow back to the high health state. This happens through a number of mechanisms. First, with a bigger budget, more treatments can be financed as healthcare rationing becomes less likely. This is both rationing at the patient side (due to co-payments) and the provider side (e.g. due to waiting lists). Second, with a higher budget it becomes more profitable for hospitals to treat patients and hence they invest more in quality to attract patients.

\subsubsection{provider choice}
\label{sec:org44a3265}

Facilitating patients to choose their own provider, increases provider competition (\(\theta\) increases) and increases the probability that a patient is cured:
\begin{equation}
\label{eq:14}
\frac{d\tau_i}{d\theta} = \frac{1}{2 \theta^2 \Delta V} > 0
\end{equation}
Hence, we expect provider choice to improve the performance of the healthcare system. Another instrument to help patients select the best hospital is to publish provider quality reports. In our model this can also be captured by an increase in \(\theta\). Since the second derivative of \(\tau_i\) with respect to \(\theta\) is negative, the analysis suggests that increasing \(\theta\) has a particular big effect on quality when \(\theta\) is low to start with. Put differently, the model expects that making quality reports public has a big effect when provider choice is restricted and a smaller impact on treatment quality when provider choice is free. This "decreasing returns to \(\theta\)" effect is indeed what we find in the data below. But an alternative hypothesis could be that free choice and public quality reports are complements: free choice has a bigger effect on quality efforts when provider quality reports are made public. This is not what we find in the data.

\cite{dranoveReportCards2003} provides another reason why the positive effects of making quality reports public are limited when there is provider choice: risk selection by providers. Physicians try to avoid treating high risk patients as it could negatively affect their score card.

\subsubsection{regulator}
\label{sec:org5d3fc05}

Another way to stimulate quality investments is to introduce a healthcare regulator which sets and enforces minimum quality standards \(\underline \tau\). If these standards are binding, imposing them leads to a direct increase in treatment quality. But there can also be an indirect effect because quality investments are strategic complements. Writing the first order condition for \(\tau_1\) without imposing a symmetric equilibrium, we see that reaction functions are upward sloping: 
\begin{equation}
\label{eq:28}
\tau_1 = \frac{R-c_s}{2\gamma}-\frac{1}{4\theta \Delta V} + \textstyle{1\over 2}  \tau_2
\end{equation}
Hence, if new standards push up \(\tau_2\), these will tend to increase \(\tau_1\) which further increases \(\tau_2\) etc. Hence, there can be a multiplier effect leading to a higher equilibrium quality increase than the increase in the standard.

\subsubsection{nurses and primary care}
\label{sec:org4508dcb}

As stated by \cite{OECD_2020}: "Strengthening primary care has been identified as an effective policy tool to improve care coordination and health outcomes and reduce wasteful spending\ldots{} However, in many EU and OECD countries, primary care has not yet fully realised this potential". Similarly, in the same report: "advanced practice nurses can improve access to services and reduce waiting times, while delivering the same quality of care as doctors for a range of patients". Hence, both primary care and the employment of nurses instead of doctors have the potential to save money which can --for given expenditure-- be used to improve healthcare and reduce mortality.

As shown in Figure \ref{fig:org225a9c0}, there is quite some variation in the nurses/doctor ratio among the countries in our data set. In some countries this ratio is close to 1 while in others it almost equals 5. Note that we only include this ratio in our regression, not the number of nurses and doctors. The reason is that we condition on healthcare expenditure which captures level effects. Hence we are interested in the question: for given expenditure, does mortality fall if five times more nurses than doctors are employed?

\begin{figure}[htbp]
\centering
\includegraphics[width=.9\linewidth]{./figures/nurses_doctors.png}
\caption{\label{fig:org225a9c0}Variation in nurses/doctors ratio across countries and time.}
\end{figure}

Increasing the use of primary care and nurses, increases the probability that a patient is cured in hospital:
\begin{equation}
\label{eq:15}
\frac{d\tau_i}{d \psi_i} = \frac{1}{\gamma} \frac{dR}{d \psi_i} > 0, i = n, pc
\end{equation}
Using healthcare resources more efficiently, creates a bigger remaining budget for healthcare (the analysis is conditional on expenditure) and hence the same mechanisms play a role as in equation \eqref{eq:16}. More money available leads to less rationing and more investments in quality. In a similar vein, cost-effectiveness analysis also helps to save resources and hence increases the remaining budget. We expect that cost effectiveness analysis increases the probability that a patient is cured and thus reduces mortality.

\subsubsection{single payer vs. competing insurers}
\label{sec:org8b78684}

Another important policy choice is whether there is a single payer in a country or competing insurers. There are a number of pros and cons of a single payer compared to a system with competing insurers. It is beyond the scope of this paper to review/model each of these. We introduce a simple bargaining model to illustrate some trade offs. In particular, we focus on the transfer of rents \(t_{ij}\) from insurer \(i\) to provider \(j\). The link to causality is that the higher this rent \(t_{ij}\), the lower the healthcare budget available for other health enhancing activities. An insurer \(i\) (the insurer in case of a single payer) offers provider \(j\) capitation fee \(t_{ij}\) and fee-for-service \(R_{ij}\) which maximize the Nash bargaining function:
\begin{equation}
\label{eq:7}
B = (\nu(R_{ij}) - R_{ij}q_{ij}(R_{ij}) - t_{ij} - \nu_{-j})^{\alpha}((R_{ij}-c_s)q_{ij}(R_{ij}) + t_{ij} - \omega_{-i})^{1-\alpha}
\end{equation}
where \(\alpha\) denotes the bargaining power of the insurer (relative to the provider), \(\nu(R_{ij})\) denotes the value to the insurer of a network including provider \(j\) receiving a fee-for-service \(R_{ij}\) who treats \(q_{ij}(R_{ij})\) of \(i\)'s customers and hence \(i\) pays \(R_{ij}q_{ij}(R_{ij})\) to \(j\) in terms of fee-for-service. Equation \eqref{eq:13} shows that an increase in \(R\) increases quality \(\tau\) which increases the value for insured if this provider is in the insurer's network. This is captured by \(\nu(R)\).

Insurer \(i\)'s outside option of not having provider \(j\) in the network is denoted by \(\nu_{-j}\); this outside option does not depend on \(R_{ij}\) because \(j\) rejects \(i\)'s contract. The net surplus for provider \(j\) is given by the fee-for-service revenue received from the insurer \(R_{ij}q_{ij}\) minus the cost of treatment \(c_s q_{ij}\) plus the capitation fee \(t_{ij}\) minus the outside option for provider \(j\) if she has no contract with insurer \(i\), \(\omega_{-i}\).

The Nash bargaining outcome is determined by \(\max_{t_{ij},R_{ij}} B\), which yields the following. The first order condition for the fee-for-service can be written as
\begin{equation}
\label{eq:21}
\nu'(R_{ij}) = c_s q'_{ij}(R_{ij})
\end{equation}
In words, the fee-for-service equalizes the marginal value for the insurer of contracting \(j\) and increasing \(R_{ij}\) to the marginal treatment cost \(c_s\) of increasing \(R_{ij}\). Note that equation \eqref{eq:21} refers to the total value and the total cost for insurer and provider combined. That is, increasing \(R_{ij}\) increases the cost for the insurer directly as well (i.e. for given \(q_{ij}\)). But this extra cost for the insurer is canceled against the extra revenue for the provider. The total cost of an increase in \(R_{ij}\) is when it leads to more treatments (\(q'_{ij}>0\)) and treatments themselves are costly (\(c_s > 0\)). 

The first order condition for the fixed fee is
\begin{equation}
\label{eq:20}
t_{ij} + R_{ij} q_{ij} = (1-\alpha) \Delta \nu+ \alpha (c_s q_{ij} + \omega_{-i})
\end{equation}
or equivalently,
\begin{equation}
\label{eq:23}
t_{ij} + R_{ij} q_{ij} = c_s q_{ij} + \omega_{-i} + (1-\alpha) (\Delta \nu - (c_s q_{ij} + \omega_{-i}))
\end{equation}
The total rent (\(t+Rq\)) transferred from the insurer to the provider equals the provider's treatment costs and outside option (that is, without insurer \(i\)) plus a share \(1-\alpha\) of the value added of provider \(j\) to \(i\)'s network: \(\Delta \nu - (c_s q_{ij} + \omega_{-i})\). The fee-for-service has efficiency effects by raising quality, but the capitation \(t\) part is just a rent transfer. This reduces total budget available for treatments and tends to reduce system performance.

We assume that only difference between a single payer (sp) and multiple insurers (mi) is that sp has more bargaining power relative to the provider than one of multiple insurers: \(\alpha^{sp} > \alpha^{mi}\). This argument is often mentioned as an advantage of a single payer system; see, for example \cite{OberlanderSinglePayer} and \cite{BICHAY2020113454}. Note that equation \eqref{eq:21} implies that \(R_{ij}\) and \(q_{ij}\) are not affected by the number of insurers. Hence, we find the following effects. If \(\Delta \nu< c_s q_{ij} + \omega_{-i}\) we find that \(t^{sp} > t^{mi}\); while \(\Delta \nu > c_s q_{ij}+ \omega_{-i}\) implies that \(t^{sp} < t^{mi}\). In words, if there are many providers, the value of an additional provider \(\Delta \nu\) is low. In an un-concentrated market, an insurer can afford not to contract some providers and contract with the others. In this case, competing insurers lead to lower rent transfers to providers, leaving more money available for other mortality reducing activities. If, on the other hand, the provider market is concentrated, the extra value of contracting one provider, \(\Delta \nu\), is high as there are few alternative providers. If \(\Delta \nu > c_s q_{ij}+ \omega_{-i}\), a single payer leads to lower transfers.

\subsubsection{public vs private providers}
\label{sec:orgfce4e3d}

We also consider the effect of organizing care either publicly or through private contracts/entities. One way to capture this distinction in the simple bargaining model here is to assume that private providers move more strategically with a closer eye on profits in case the contract with an insurer falls through. That is, the outside option is higher for private than public providers: \(\omega_{-i}^{pr} > \omega_{-i}^{pu}\). Since \(\alpha^{sp}>\alpha^{mi}\), the weight on \(\omega_{-i}\) is higher for a single payer. Hence, a single payer benefits more from a public provider (compared to private provider) than competing insurers.

Taking the latter interaction into account, we find that a single payer is particularly effective in concentrated markets with publicly organized providers. If markets are un-concentrated, competing insurers are more effective and the distinction between public/private providers is less of an issue as the weight \(\alpha^{mi}\) on \(\omega_{-i}\) is relatively small.

\subsubsection{effect of income on treatment quality}
\label{sec:orga98123f}

Finally, we consider the effect of income \(Y\) on the quality of treatment \(\tau\). From equation \eqref{eq:13} we find that
\begin{equation}
\label{eq:29}
\frac{d\tau_i}{dY} = \frac{1}{2\theta (\Delta V)^2}\frac{d \Delta V}{dY} > 0
\end{equation}
because of \eqref{eq:26}. As income per head increases, the quality difference between hospitals has a bigger utility effect and hence patients choose their provider more carefully. This raises hospitals' incentives to invest in quality. For given healthcare expenditure, richer countries (higher GDP per capita) have more patients that are sensitive to provider quality. This creates higher incentives for hospitals to invest in quality thereby improving the performance of the healthcare system.

The second derivative of \(\tau_i\) with respect to \(Y\) can be written as
\begin{equation}
\label{eq:30}
\frac{d^{2}\tau_i}{dY^{2}} = \frac{-1}{\theta (\Delta V)^3} \left(\frac{d \Delta V}{dY}\right)^2 +\frac{1}{2\theta (\Delta V)^2} \frac{d^2 \Delta V}{dY^2}
\end{equation}
A sufficient condition for \(d^2 \tau_i/dY^2 <0\) is that \(d^2 \Delta V/dY^2 \leq 0\). Writing out the complete derivative and signing all the terms is beyond the scope of this paper. But a quick look at equation \eqref{eq:26} suggests that we need something like: an increase in \(Y\) reduces the numerator (\(\partial^2 y^h/\partial Y^2 -\partial^2 y^l/\partial Y^2<0\). This is the case if utility is additive in consumption and of the form \(Y^{\xi}\) in the high state and \(\kappa Y^{\xi}\) in the low health state with \(\xi,\kappa \in \langle 0,1\rangle\). 

The significance of \(d^2 \tau_i /dY^2 <0\) is the following. If a provider faces a population of agents with differing incomes, we need to integrate over this population to determine the aggregate incentive for hospitals to invest in quality. If quality \(\tau_i\) is concave in income this implies that quality is higher --for given average income-- if the variance in income is lower (Jensen's inequality for a concave function). This gives a micro foundation for the observation in equation \eqref{eq:3} that mortality increases with income inequality. Similar assumptions can be made for \(\zeta\) to ensure that the aggregate \(\zeta\) is lower --the population healthier-- as income inequality is lower.

We summarize the discussion above as follows.

\begin{itemize}
\item Mortality is affected by:
\begin{itemize}
\item \textbf{health factors}:  income per head, \(Y\), income inequality, gender and lifestyle choices such as smoking, alcohol consumption and high bmi;
\item \textbf{expenditure}: healthcare expenditure;
\item \textbf{efficiency}: cost effectiveness analysis for the introduction of new treatments, the use of nurses (compared to doctors/specialists) and how well the primary care sector is developed;
\item \textbf{quality policies}: making provider quality reports public, presence of a healthcare regulator;
\item \textbf{market policies}: provider choice, single payer, whether care services are publicly or privately organized;
\item \textbf{interaction effects}: the effect of gender on mortality is affected by income per head, effect of publicly available quality reports is affected by free provider choice, effect of single payer is affected by whether providers are predominantly publicly/privately organized.
\end{itemize}
\end{itemize}

\subsection{Causality}
\label{sec:org37e5ae8}

The main question of the paper is to see whether the theoretical effects on the performance of healthcare systems described above are in line with the data that we have on system performance across countries. Hence, we check whether the correlations implied by the model are strong enough to be detected in the macro data. We do not try to find instruments for each policy measure to prove causality, we just verify whether the predictions of the model are borne out by the data.

For some variables it is clear that we do not identify causal effects. To illustrate, a positive correlation between income and health can have different causal links. Higher income leads to healthier choices and hence lower mortality. But a population in better health is more productive and hence generates a higher income is also possible. For our controlling variables, we are not interested in the underlying causal model.

For the following variables we are interested in the causal relation with health/mortality: ratio of nurses to doctors, primary care, the use of CE analysis, whether patients can choose their providers, single payer vs. multiple insurers, public vs private providers, regulator dedicated to healthcare and whether provider quality reports are made public. For these reversed causality does not really seem to be an issue. Lower mortality leading to the introduction of a regulator is not an intuitive mechanism. Self selection is also not obvious. To illustrate, we find in the data that free provider choice leads to lower mortality. One could argue that countries that know that free provider choice will reduce mortality implement this policy while countries where free choice does not work, refrain from this policy. This cannot be ruled out. But we are not aware of results that suggest that free provider choice works for some countries and not for others. Many policy makers doubt whether free provider choice works at all to improve the performance of their healthcare system. In a sense, the goal of this paper is to get some more clarity about the effects of different policies on system performance.

An effect that we cannot exclude is that the variables mentioned above are correlated with policies that we do not control for. For example, countries that use CE analysis also feature a relatively high performance bonus for physicians. It can be the case that the effect on mortality is caused by performance contracts but we attribute the effect to CE analysis. Since our data set is relatively small, we cannot control for all different policies. Hence we cannot exclude this possibility; however, we do focus on policies that are supposed to have major effects on health outcomes and therefore are likely to dominate other effects.


\section{Data}
\label{sec:orgb0fa0da}

The objective of the analysis is to compare the functioning of healthcare systems between different countries with different institutional features. Finding data measured in a consistent way across countries is not obvious but recently Eurostat (\url{https://ec.europa.eu/eurostat/data/database}) and the OECD (\url{https://qdd.oecd.org/subject.aspx?Subject=hsc}) have made progress in this direction. Combining both sources, we have data on health statistics (like mortality, smoking behavior, expenditure etc.) and on healthcare systems (does a country have free hospital choice, a healthcare regulator, make provider quality reports publicly available etc.) that are collected in a consistent way and thus are comparable between countries. We use this data to analyze the performance of healthcare systems for the 24 countries in Table \ref{tab:orgd6eb02a}.

\begin{table}[htbp]
\caption{\label{tab:orgd6eb02a}Four variables grouped by country}
\centering
\begin{tabular}{lrrrr}
Country & treatable & HC1 & avoidable & nurses/doctor\\
 & mortality &  & hospital. & ratio\\
\hline
Switzerland & 56.81 & 3775.40 & 126.05 & 4.04\\
Iceland & 63.34 & 1925.93 & 200.48 & 4.29\\
France & 64.82 & 1584.12 & 144.55 & nan\\
Norway & 68.23 & 3231.76 & 259.67 & 4.00\\
Spain & 70.12 & 1196.45 & 222.23 & nan\\
Italy & 71.14 & nan & 74.08 & 1.46\\
Sweden & 72.05 & nan & 187.63 & 2.85\\
Netherlands & 72.16 & 1918.72 & 227.43 & 3.08\\
Luxembourg & 75.08 & 2544.74 & 198.65 & 4.23\\
Belgium & 77.02 & 1654.62 & 240.02 & 3.70\\
Austria & 79.38 & 2121.48 & 265.93 & 1.39\\
Finland & 80.33 & 2012.29 & 187.65 & 4.71\\
Denmark & 80.70 & 2657.79 & 326.90 & 2.63\\
Ireland & 84.55 & nan & 402.28 & nan\\
Germany & 90.20 & 1924.87 & 282.90 & 3.13\\
United Kingdom & 90.76 & nan & 298.07 & 3.07\\
Slovenia & 90.89 & 902.98 & 139.82 & 3.21\\
Portugal & 90.96 & 964.57 & 84.23 & nan\\
Greece & 95.60 & 851.08 & nan & nan\\
Poland & 140.88 & 411.54 & 246.46 & 2.51\\
Estonia & 160.11 & 519.07 & 104.52 & 1.88\\
Hungary & 191.53 & 380.92 & 427.50 & 2.08\\
Lithuania & 222.60 & 393.38 & 291.57 & 1.84\\
Latvia & 228.19 & 337.67 & 251.97 & 1.56\\
\end{tabular}
\end{table}

Finding a summary measure to compare or rank healthcare systems is not straightforward. Here we use an obvious but not very subtle variable: mortality. On the one hand, a goal of any healthcare system is to prevent people from dying prematurely. However, mortality measures ignore issues related to quality of life which are equally important but harder to measure. The goal of healthcare is not as long a life as possible, but high quality of life. Not every disease can be cured, caring for such patients in a respectful manner is a key part of healthcare but does not necessarily reduce mortality (by much).

However, comparable data on quality of life (e.g. via quality adjusted life years --qaly's) are hard to collect across countries. The problem that mortality is not representative of healthcare in general is most acute with measures that are specific like expenditure in a particular category. This is one of the reasons why in our analysis we will control for expenditure but not view it as a policy variable. This avoids silly policy recommendations like shift expenditure from long term care and spend it on prevention.

However, with generic policy measures like introducing a healthcare regulator it is harder to see why this would reduce mortality but not the quality of healthcare more generally. Another advantage of the mortality measures that we use is that they are based on a standardized age distribution. Hence, we do not need to control for age in our model. For instance, because a population with a high fraction of elderly can be expected to have higher mortality. In fact, with a standardized mortality measure, the opposite is the case as illustrated in Figure \ref{fig:org8710a6c}. Populations with a high median age tend to have low (standardized) mortality, otherwise the median age would not be that high.

\begin{figure}[htbp]
\centering
\includegraphics[width=.9\linewidth]{./figures/age.png}
\caption{\label{fig:org8710a6c}The relation between median age in the population and a standardized mortality measure.}
\end{figure}

Appendix \ref{sec:mortality} provides details on the mortality measures. Here we provide the definitions and illustrate these with examples. In the main text we work with treatable mortality, for two robustness analyses we work with preventable mortality and Potential Years of Life Lost (PYLL). An advantage of PYLL is that we can evaluate the benefit of certain policies by choosing a value for a life year. This is not obvious to do with a mortality measure as we cannot differentiate between death avoided at age 20 and age 60.

Treatable mortality refers to causes of death that can be mainly avoided through timely and effective health care interventions, including secondary prevention such as screening, and treatment after the onset of disease. Preventable mortality is defined as causes of death that can be mainly avoided through effective public health and primary prevention interventions before the onset of diseases/injuries \citep{countryprofileUK}. To illustrate, breast cancer is labeled as treatable but not preventable, lung and bladder cancer are preventable (reduce smoking) but not treatable and hypertensive diseases are both preventable (reduce smoking, improve nutrition and physical activity) and treatable. Death rates are expressed per 100,000 inhabitants using a weighted average of age-specific mortality rates, where the weights are based on the age distribution of a standard reference population.

PYLL is an indicator estimating the potential years lost due to premature death, i.e. death before 70. It is calculated by summing the number of years between the age at death and 70 years for each premature death. PYLL rate is expressed per 100,000 age-standardised population under 70.\footnote{\url{https://ec.europa.eu/eurostat/cache/metadata/en/hlth\_cdeath\_esms.htm}}

We choose treatable mortality as our variable to be explained in the main text because it is more focused on healthcare policy than PYLL (or other broader mortality measures) and includes both prevention measures and treatments. Preventable mortality is more focused on health policy in the area of prevention and less on the cure of diseases, say, in hospitals.


\subsection{Missing values}
\label{sec:orgac0c417}

Table \ref{tab:orgd6eb02a} presents country averages for four variables for the countries in our data ordered on treatable mortality. Switzerland has lowest (treatable) mortality in our European sample and Latvia the highest. This can be due to the fact that the Swiss have organized their healthcare particularly well, but an important explanation is that the Swiss have high healthcare spending per capita. The HC1 column (curative care) shows that this is indeed the case for this healthcare spending category. Also  avoidable hospitalizations and the nurses/doctor ratio are presented.

The table highlights one of the challenges for this paper: missing values (denoted "nan", not a number). Although steps have been made by Eurostat and the OECD to gather consistent data on health, the data are not complete. Deleting observations with a missing value on a variable is not an option as too many observations will be lost. Hence we need to find a robust way to deal with missing values. This is straightforward when using Bayesian estimation.

In Bayesian analysis there is a natural way to deal with missing variables which is an improvement on two standard ways of dealing with this: (i) dropping observations with missing values (sometimes called complete case analysis) and (ii) interpolating the missing values. Dropping observations is an option we cannot afford. As the consistently measured health data that we use here is quite recent, there are not that many observations. Dropping observations will make inference close to impossible. Interpolating data, say by replacing a missing value with the mean value of the variable makes the estimation method "too confident" about this value, thereby negatively affecting the quality of the inference.

We use the following method to deal with missing values \citep{mcelreath}. In our data we only have missing values for a number of independent variables; not for the mortality variables that we use. The following continuous variables feature missing values: nurses/doctor ratio, income inequality, avoidable hospitalizations, smoking, alcohol consumption, BMI, and the seven healthcare expenditure categories. Since we standardize these variables in our Bayesian analysis, we know that their values are distributed with mean 0 and standard deviation equal to 1. If the variable has a time dimension, we allow the mean to vary by country. That is, if a country has a high nurses/doctor ratio in one year, it is likely to have a high value the next year. Further, as these variables are based on sums, we assume that they are normally distributed.

The uncertainty surrounding the value of a missing observation is taken into account in the posterior distributions of our parameters. When sampling the posterior, if we encounter a missing value, this value is drawn from its distribution. We work with 2000 samples and hence we draw 2000 different values for each missing observation. In this way, the uncertainty about the value translates into posterior uncertainty of the parameters and predictions.

A different but related issue is that not all of our variables vary with sex, country and year. To illustrate, GDP per head does not vary with sex in our data and our health system data does not vary by sex nor by year. Section \ref{sec:data_dimensions} shows for each of our variables over which dimensions they vary. For our countries the OECD has only one observation about their health system. For all countries but one this OECD data is for 2016. Only for Hungary do we have system data for 2012. We use \cite{countryprofileReport} and the accompanying country reports to verify that there have been no further reforms in our data period.


\subsection{Variables}
\label{sec:org6b9ce11}

Here we discuss each of the variables used in our analysis. We have data for the years 2011-2017, 24 countries and two genders. Except for France where we miss mortality for the year 2017. Hence, we have \(7*2*24-2=334\) rows in our data frame. Table \ref{tab:org97c867f} summarizes the data for the continuous variables. The first three rows are the mortality variables discussed above.

\begin{table}[htbp]
\caption{\label{tab:org97c867f}Summary statistics continuous variables}
\centering
\begin{tabular}{lrrrrr}
 & count & mean & std & min & max\\
\hline
treatable mortality (per 100,000) & 334.00 & 100.94 & 57.18 & 49.32 & 333.31\\
preventable mortality (per 100,000) & 334.00 & 186.27 & 130.40 & 52.94 & 660.16\\
potential years of life lost (per 100,000) & 334.00 & 3887.24 & 2323.70 & 1664.00 & 13271.00\\
avoidable hospitalizations (per 100,000) & 226.00 & 225.65 & 84.22 & 36.30 & 427.50\\
log GDP per capita & 334.00 & 10.28 & 0.61 & 9.20 & 11.46\\
inequality & 278.00 & 0.16 & 0.05 & 0.12 & 0.29\\
smoking & 320.00 & 0.07 & 0.05 & 0.00 & 0.25\\
high bmi & 320.00 & 0.15 & 0.03 & 0.08 & 0.21\\
alcohol & 294.00 & 0.20 & 0.11 & 0.03 & 0.57\\
nurses/doctor ratio & 244.00 & 2.90 & 1.00 & 1.37 & 4.85\\
\end{tabular}
\end{table}

Avoidable hospitalizations refer to the number of hospital admissions that could have been dealt with at the primary care level. "A key aim of primary care is to keep people well by providing a consistent point of care over the longer term, treating the most common conditions, tailoring and co-ordinating care for those with multiple health care needs and supporting the patient in self-education and self-management" \citep{OECD_avoidable_hospitalizations}. If primary care works well, patients do not need to visit a hospital for treatment for a number of conditions. Our measure is based on asthma and COPD hospital admissions in adults which can be kept low (but not necessarily zero) through well working primary care by preventing an acute deterioration in patients with these chronic conditions. We interpret low avoidable hospitalizations as implying well organized primary care where gate keeping prevents patients from visiting hospitals when this is not necessary and where coordination of services prevent the need to visit a hospital in the first place. There is substantial variation with only 36 hospital admissions in one country and 428 in another (per 100,000 population).

The avoidable hospitalizations measure is preferable to a variable measuring expenditure on primary care or a dummy that just measures whether there is primary care gate keeping. Comparing primary care expenditure across countries is hard because a uniform definition of which services constitute primary care is missing \citep{OECD_2020}. A problem with a gate keeping dummy is that it can be introduced to deal with a lack of healthcare resources and as such does not improve performance. Avoidable hospitalizations focuses more on the potential strength of the primary care sector and we analyze whether this has measurable macro effects.

Log GDP per capita is available for all our observations but inequality measured as in equation \eqref{eq:19} is not always available. This is because the income quantiles on which this measure is based are not in our data for each country-year combination.

Smoking refers to the fraction of the population (aged 15-64, by gender) which smokes 20 or more cigarettes a day. High BMI refers to the fraction of the population (again, aged 15-64 by gender) with a body mass index above 30. This is usually referred to as obese. Alcohol refers to the incidence (fraction of 15-64 year olds by gender) of heavy episodic drinking (binge drinking) at least once a month during 12 months. That is, ingesting more than 60g of pure ethanol (6 units of alcohol) on a single occasion. Binge drinking tends to happen more often than smoking more than 20 cigarettes per day or being obese.

Finally, we consider the mix of nurses and doctors in the health production function. In the past years, the role of nurses has increased considerably, taking over more and more tasks that were traditionally done by doctors. One can think here of an increased role in the management of patients with a chronic condition and dealing with patients with relatively minor health problems. As explained in the theory section, we view this as a way to increase efficiency.

Of the variables in Table \ref{tab:org97c867f}, we view avoidable hospitalizations and the nurses/doctor ratio as health policy variables. If these variables tend to be associated with low mortality, we would suggest governments to improve their primary care sector to reduce avoidable hospitalizations and to increases the number of nurses compared to doctors.

Income per head and income inequality are viewed as variables controlling for lifestyle, together with the direct lifestyle variables related to smoking, obesity and alcohol consumption. As one would expect, reducing the incidence of smoking etc. reduces mortality, but this is not the focus of this paper. We only want to control for these effects when considering our policy variables.

The binary policy variables are summarized in Table \ref{tab:orgf7df7d1}. Whether cost effectiveness --next to efficacy-- plays a role in allowing new treatments to be covered by health insurance, which most countries do. Whether a country has a national health system covering the country (like the NHS in the UK) or the country features a single health insurance fund. Both are captured by the variable single payer system. We also consider whether primary, inpatient and outpatient care is predominantly done through publicly employed physicians or public hospitals instead of private ones. The variable healthcare regulator captures whether there is an organisation with responsibility for national policy on health care quality in the country. The variable public provider quality reports refers to whether or not quality metrics are reported publicly at the provider level at least annually. Hospital choice refers to the cases where patients are completely free to choose their hospital and they are free to choose but face financial incentives to choose some hospitals (rather than others). In either case, patients can "vote with their feet" in case a hospital has a bad reputation. Inpatient care refers to the treatment and/or care provided in a healthcare facility to patients formally admitted and requiring an overnight stay. Outpatient care refers to medical and ancillary services delivered in a healthcare facility to a patient who is not formally admitted and does not stay overnight. The last three variables in the table refer to whether the relevant form of care is predominantly supplied through public (instead of private) providers.

For these variables the column mean gives the fraction of countries for which there is cost effectiveness analysis, single payer etc.

\begin{table}[htbp]
\caption{\label{tab:orgf7df7d1}Summary statistics binary variables}
\centering
\begin{tabular}{lrrr}
 & count & mean & std\\
\hline
ce analysis & 334.00 & 0.96 & 0.20\\
single payer system & 334.00 & 0.75 & 0.43\\
healthcare regulator & 334.00 & 0.83 & 0.37\\
public provider quality reports & 334.00 & 0.50 & 0.50\\
hospital choice & 334.00 & 0.83 & 0.37\\
public primary care & 334.00 & 0.38 & 0.49\\
inpatient public & 334.00 & 0.71 & 0.46\\
outpatient public & 334.00 & 0.50 & 0.50\\
\end{tabular}
\end{table}


Finally, we control for healthcare expenditures per head in seven categories; see Table \ref{tab:orgc24a9ee}. As shown in Table \ref{tab:org7a5d935}, expenditure per head is highest for curative care (HC1) and long-term care (HC3) is next highest. We will not interpret the coefficients on these variables as policy recommendations. Given that our dependent variable is mortality, one can expect the effect of HC1 to be bigger than HC3, but this does not imply that HC3 expenditure is inefficient. Long-term care is an important part of health policy but its main effect is not to reduce mortality.

\begin{table}[htbp]
\caption{\label{tab:orgc24a9ee}Expenditure variables}
\centering
\begin{tabular}{ll}
code & function\\
\hline
HC1 & Curative care\\
HC2 & Rehabilitative care\\
HC3 & Long-term care (health)\\
HC4 & Ancillary services (non-specified by function)\\
HC5 & Medical goods (non-specified by function)\\
HC6 & Preventive care\\
HC7 & Governance and health system and financing administration\\
\end{tabular}
\end{table}


\begin{table}[htbp]
\caption{\label{tab:org7a5d935}Summary statistics expenditure variables (euro per head)}
\centering
\begin{tabular}{lrrrrr}
 & count & mean & std & min & max\\
\hline
HC1 & 264.00 & 1616.47 & 987.00 & 289.20 & 4247.68\\
HC2 & 250.00 & 131.62 & 122.06 & 0.86 & 412.50\\
HC3 & 314.00 & 632.09 & 538.93 & 8.34 & 1988.62\\
HC4 & 314.00 & 151.57 & 109.51 & 29.96 & 537.71\\
HC5 & 314.00 & 526.03 & 217.68 & 158.50 & 1232.15\\
HC6 & 314.00 & 89.50 & 58.32 & 4.44 & 213.90\\
HC7 & 314.00 & 92.03 & 79.55 & 10.22 & 339.94\\
\end{tabular}
\end{table}

In order to facilitate the interpretation of the coefficients on the spending categories, we write the equation to be estimated as follows.
\begin{align}
\label{eq:17}
\ln(M) &= ... + \sum_{i=1}^7 b_{hc_i} \ln(HCi) \\
\label{eq:18}
 &= ... + \left(\sum_{i=1}^7 b_{hc_i} \right) \ln(HC) +  \sum_{i=1}^7 b_{hc_i} \ln(HCi/HC) 
\end{align}
where \(HC =  \sum_{i=1}^7 HCi\) denotes total expenditure across all categories. Hence, the sum of the coefficients \(b_{hc_i}\) captures the effect of increasing total expenditure for given fractions \(HCi/HC\) and individual coefficients \(b_{hc_i}\) capture the effect of changing the distribution of expenditure \(HC\) over the different categories \(i\). In the tables below, we will also report on the posterior distribution of \(b_{hc}\), although this is not a separately estimated coefficient.

Note that we work here with log expenditure per head. Another option would be to define healthcare expenditure as a fraction of GDP. Since log GDP per head is also in our regression, these two specifications are equivalent in our case since we can write \(\alpha \ln(HC/Y) + \beta \ln(Y) = \alpha \ln(HC) + (\beta-\alpha) \ln(Y)\). As we want to condition our results on these variables but are not specifically interested in their coefficients, for us the results are the same whether we work with health expenditure per head or fraction of GDP spent on health.




\section{Structural model}
\label{sec:org4869315}

In this section we describe the model that we actually estimate on the data described above.  All continuous variables are standardized (mean subtracted and divided by the standard deviation). Just to be clear, when using logs, we first take the log and then we standardize.

We estimate the following model:
\begin{align*}
\ln(M_{jtg}) &\sim N(\mu_{jtg},\sigma) \\
\sigma & \sim\text{HalfNormal}(1) \\
\mu_{jtg} &= \sum_{i=1}^7 b_{hci} \ln(\text{HCi}_{jt}) + b_{\text{female}} I_{\text{female}_{jtg}} + b_{\text{gdp}} \ln(\text{GDP per head}_{jt}) \\
 & + b_{\text{female\_gdp}} \ln(\text{GDP per head}_{jt})  I_{\text{female}_{jtg}} \\
 & + b_{\text{inequality}} \text{inequality}_{j} + b_{\text{smoking}} \text{smoking}_{jg}+ b_{\text{high bmi}} \text{high bmi}_{jg}+ b_{\text{alcohol}} \text{alcohol}_{jg} \\
 & + b_{\text{nurses\_doctors}} \text{ratio nurses/doctors}_{jt} + b_{\text{avoidable\_hospitalizations}} \text{avoidable hospitalizations}_{jt} \\
 & + b_{\text{ce\_analysis}} I_{\text{ce analysis}_{j}} +  b_{\text{hospital\_choice}} I_{\text{hospital choice}_{j}} + b_{\text{single\_payer}} I_{\text{single payer}_{j}} \\
 & +  b_{\text{public\_primary\_care}} I_{\text{public primary care}_j} + b_{\text{inpatient\_public}} I_{\text{inpatient public}_{j}}+ b_{\text{outpatient\_public}} I_{\text{outpatient public}_j} \\
 & + b_{\text{healthcare\_regulator}} I_{\text{healthcare regulator}_j} + b_{\text{quality\_reports}} I_{\text{quality reports}_j} \\
 & + b_{\text{reports\_no\_choice}} I_{\text{quality reports}_j} (1-I_{\text{hospital choice}_j})
\end{align*}
where we index on country \(j\), year \(t\) and gender \(g\). We assume that log mortality is normally distributed with mean \(\mu_{jtg}\) and standard deviation \(\sigma\). Overall \(\ln(M)\) is standardized and has standard deviation equal to 1. However, different observations have a different mean which adds to the standard deviation of the predicted \(\ln(M)\). Hence, we assume that \(\sigma\) is half-normally distributed with mean \(\sqrt{2/\pi}\) (parameter 1 in the half-normal distribution). Expected log mortality is then a function of the independent variables introduced above, where \(I\) denotes the indicator function (dummy variable) equal to 1 if gender is female, if a country has a healthcare regulator etc. We have two interaction effects: the gender mortality gap is allowed to fall with GDP per head and we consider the effect of making provider quality reports public at least once a year for countries where there is no free hospital choice.

In the main text we use treatable mortality for \(M_{jtg}\).

As this is a Bayesian analysis, we set priors for our parameters. Since the size of our data is limited, we choose conservative priors --zero expectation and relatively small standard deviations-- to avoid over-fitting  \citep{mcelreath}. This implies that the sign of our estimates tends to be more reliable than the size of the coefficients. A priori, parameters are assumed to be normally distributed with mean zero and standard deviation equal to 0.1. That is one standard deviation change in a continuous variable has an effect on log mortality of between -0.2 to 0.2 standard deviation with 95\% probability. For a dummy variable, the switch from 0 to 1 is expected to have this effect with 95\% probability.

Missing values for variables without a time dimension are assumed to be drawn from a normal distribution with mean 0 and standard deviation equal to 1 (as these variables are standardized). If the variable has a time dimension, we allow the mean to vary with country. To illustrate, if the country has high realizations of the variables for a number of years, we do not assume that the missing value is drawn from a distribution with mean zero. The mean of the distribution will be strictly positive in this case. But we do assume that the standard deviation equals 1 to make sure that sufficient uncertainty remains in the posterior distribution.


\section{Results}
\label{sec:org2079f8a}

Here we present the main characteristics of the posterior distribution of our model. Before looking at the results, let's get an idea of the fit of the model.

\subsection{Model fit}
\label{sec:org07c8de3}

Figure \ref{fig:orgf9a24dd} plots the data (points) as in Figure \ref{fig:org5829a41} (using the standardized variables we use in the model) together with the posterior predictive distribution. The line-segments present the 95\% interval of our predictions for each of our observations.

That is, each estimated parameter has a distribution (see below) and each point in the graph represents an observation of a country in a certain year for a given gender. We fill in all the variable values we have for that observation and multiply these values with the whole posterior distribution of the relevant parameter. If we have a missing observation for a variable, we draw from the distribution of this variable. Then we add all these distributions to get to the distribution of (standardized) log mortality. The line-segments indicate 95\% of the observations in this posterior distribution for log mortality. All observations (points) fall within their 95\% predictive interval (line-segment). In this sense, the fit is quite reasonable.


\begin{figure}[htbp]
\centering
\includegraphics[width=.9\linewidth]{./figures/check_fit.png}
\caption{\label{fig:orgf9a24dd}Model fit in terms of log treatable mortality and log GDP per head.}
\end{figure}

\subsection{main specification}
\label{sec:org88414ab}

\begin{table}[htbp]
\caption{\label{tab:org2052a02}Summary posterior distributions of the model's parameters}
\centering
\begin{tabular}{lrrrrrr}
coefficient & mean & sd & hpd\_3\% & hpd\_97\% & ess\_mean & r\_hat\\
\hline
b\_hc & -0.537 & 0.103 & -0.737 & -0.347 & 2929 & 1\\
b\_hc1 & -0.107 & 0.025 & -0.155 & -0.062 & 2299 & 1\\
b\_hc2 & -0.008 & 0.03 & -0.063 & 0.049 & 1006 & 1\\
b\_hc3 & -0.046 & 0.053 & -0.141 & 0.056 & 2501 & 1\\
b\_hc4 & -0.095 & 0.036 & -0.163 & -0.029 & 2242 & 1\\
b\_hc5 & -0.003 & 0.043 & -0.085 & 0.074 & 2642 & 1\\
b\_hc6 & -0.117 & 0.038 & -0.193 & -0.05 & 3546 & 1\\
b\_hc7 & -0.073 & 0.036 & -0.141 & -0.007 & 3017 & 1\\
b\_female & -0.347 & 0.046 & -0.431 & -0.259 & 3669 & 1\\
b\_gdp & -0.322 & 0.063 & -0.441 & -0.206 & 3808 & 1\\
b\_female\_gdp & 0.22 & 0.03 & 0.163 & 0.275 & 5117 & 1\\
b\_inequality & 0.288 & 0.027 & 0.238 & 0.339 & 1948 & 1\\
b\_smoking & 0.134 & 0.031 & 0.077 & 0.193 & 3664 & 1\\
b\_obese & 0.051 & 0.019 & 0.015 & 0.087 & 3919 & 1\\
b\_alcohol & 0.259 & 0.03 & 0.203 & 0.318 & 2314 & 1\\
b\_nurses\_doctors & -0.138 & 0.019 & -0.173 & -0.101 & 1768 & 1\\
b\_avoidable\_hospitalizations & 0.127 & 0.016 & 0.095 & 0.156 & 2446 & 1\\
b\_ce\_analysis & -0.288 & 0.076 & -0.428 & -0.143 & 4299 & 1\\
b\_hospital\_choice & -0.446 & 0.058 & -0.556 & -0.336 & 2777 & 1\\
b\_single & -0.077 & 0.062 & -0.197 & 0.038 & 2189 & 1\\
b\_public\_primary\_care & -0.1 & 0.05 & -0.192 & -0.006 & 1935 & 1\\
b\_inpatient\_public & -0.323 & 0.052 & -0.42 & -0.227 & 3104 & 1\\
b\_outpatient\_public & 0.17 & 0.062 & 0.057 & 0.291 & 2242 & 1\\
b\_healthcare\_regulator & -0.154 & 0.053 & -0.252 & -0.056 & 3823 & 1\\
b\_quality\_reports & -0.034 & 0.055 & -0.139 & 0.065 & 2559 & 1\\
b\_reports\_no\_choice & -0.291 & 0.087 & -0.455 & -0.129 & 2533 & 1\\
\end{tabular}
\end{table}


Table \ref{tab:org2052a02} summarizes the posterior distributions of the estimated model. For all parameters r hat is equal to 1. The appendix shows the posterior plots for the parameters. The trace plots look sufficiently mixed and random that we can trust the sampling process.

We find that total healthcare expenditure reduces mortality. The signs of the parameters for the different expenditure categories indicate which of these are more effective in reducing treatable mortality as explained in equation \eqref{eq:18}. Curative care (HC1), ancillary services (HC4), preventive care (HC6) and administration expenditure (HC7) have mortality reducing effects with high probability.

On average, women have lower mortality than men and GDP per head reduces mortality. The gender gap falls with GDP per head. Income inequality within a country tends to increase treatable mortality. Note that the effect of GDP per head is not that poorer countries spend less on healthcare, since the results here are conditional on healthcare expenditure. We interpret the effects of gender and income more in terms of lifestyle effects. The explicit lifestyle variables smoking, high BMI and alcohol consumption tend to increase mortality, as one would expect. Income inequality picks up effects beyond these obvious factors. To illustrate, as income and healthcare expenditure are positively correlated (health is a normal good), higher income inequality implies that a given expenditure level per head is more skewed towards high incomes. If there are decreasing returns to healthcare expenditure in terms of mortality, higher income inequality implies higher mortality for \emph{given} expenditure per head. This is captured by equation \eqref{eq:3}. As suggested by equation \eqref{eq:30}, if quality is concave in income, higher income inequality leads to lower treatment quality.

Efficiency measures do have clear effects on mortality. Increasing the ratio of nurses to doctors saves money without directly increasing mortality. This money can then be used to reduce mortality. Of course, it is also possible that the use of nurses is more effective (i.e. not only more efficient) than doctors thereby reducing mortality directly, but the model above shows that this assumption is not necessary to explain the result. Ineffective primary care signaled by high avoidable hospitalizations tends to raise mortality. Money can be saved by having GP's resolve fairly standard problems and "gatekeep" expensive hospital care. Recall that avoidable hospitalizations do not refer to medical problems that should not have happened; the reference is to treatments that could have been provided in primary care thereby avoiding expensive hospital treatment. As we do the analysis, controlling for expenditure, the money saved in this way can be spent in other ways to reduce mortality. In the same vein, testing new treatments not only on effectiveness but also on cost efficiency before approving their use raises the performance of the health system. A commitment to a cost efficiency test not only avoids spending money on treatments where alternatives provide more value for money, it can also force inventors of new treatments (e.g. pharmaceutical companies) to charge lower prices.

Provider competition induced by allowing patients to choose their hospital reduces mortality. As suggested by equation \eqref{eq:14}, more intense provider competition leads to higher investments in treatment quality to attract patients. But on the insurer side, having a single payer instead of insurer competition tends to reduce costs and hence for given expenditure decreases mortality. However, in general, the standard deviation of the posterior distribution of \texttt{b\_single} is quite high. Below we analyze whether a single payer works better if providers are public instead of private agents. In the table here we see that having primary and inpatient care primarily arranged with public providers reduces mortality. But outpatient care is better arranged via private providers.

We consider two direct quality measures: having a healthcare regulator and making provider quality reports public. Introducing a health specific regulator tends to reduce mortality. Making quality reports public has little additional effect if there is provider choice already. But if there is no or hardly any provider choice, making these reports public does reduce mortality.

Instead of provider choice requiring public quality reports to work, we find that these measures are substitutes. This suggests that allowing provider choice generates its own information for competition to work. An alternative interpretation is that with provider choice, quality transparency can lead to risk selection by hospitals to game the system \citep{dranoveReportCards2003}. This reduces the efficacy of public quality reports. With our macro (cross country) approach, we cannot distinguish which of these mechanisms dominate.

\subsection{extended model}
\label{sec:orga680039}

To get a better understanding where the effects of public/private primary, outpatient and inpatient care come from, we extend the main model with three interaction terms. We interact single payer status of a country with the variables capturing whether primary, outpatient and inpatient care are mainly provided through public entities. Note that a single payer system does not imply that all healthcare is provided by the government.

\begin{table}[htbp]
\caption{\label{tab:org73615e2}Summary posterior distributions of the extended model's parameters}
\centering
\begin{tabular}{lrrrrrr}
coefficient & mean & sd & hpd\_3\% & hpd\_97\% & ess\_mean & r\_hat\\
\hline
b\_hc & -0.55 & 0.104 & -0.744 & -0.364 & 2599 & 1\\
b\_hc1 & -0.079 & 0.028 & -0.131 & -0.027 & 1918 & 1\\
b\_hc2 & 0.002 & 0.027 & -0.049 & 0.052 & 748 & 1.01\\
b\_hc3 & -0.034 & 0.05 & -0.129 & 0.059 & 2834 & 1\\
b\_hc4 & -0.121 & 0.035 & -0.185 & -0.053 & 1968 & 1\\
b\_hc5 & -0.01 & 0.044 & -0.093 & 0.07 & 2549 & 1\\
b\_hc6 & -0.119 & 0.037 & -0.187 & -0.048 & 3047 & 1\\
b\_hc7 & -0.084 & 0.035 & -0.149 & -0.016 & 2676 & 1\\
b\_female & -0.349 & 0.043 & -0.43 & -0.266 & 3550 & 1\\
b\_gdp & -0.358 & 0.062 & -0.477 & -0.241 & 3195 & 1\\
b\_female\_gdp & 0.225 & 0.029 & 0.172 & 0.28 & 4061 & 1\\
b\_inequality & 0.289 & 0.026 & 0.24 & 0.338 & 2083 & 1\\
b\_smoking & 0.121 & 0.029 & 0.066 & 0.174 & 3373 & 1\\
b\_obese & 0.053 & 0.019 & 0.016 & 0.086 & 3869 & 1\\
b\_alcohol & 0.271 & 0.029 & 0.216 & 0.326 & 2196 & 1\\
b\_nurses\_doctors & -0.11 & 0.019 & -0.147 & -0.075 & 1891 & 1\\
b\_avoidable\_hospitalizations & 0.106 & 0.016 & 0.077 & 0.137 & 2105 & 1\\
b\_ce\_analysis & -0.254 & 0.076 & -0.397 & -0.114 & 3763 & 1\\
b\_hospital\_choice & -0.424 & 0.055 & -0.527 & -0.319 & 2835 & 1\\
b\_single & 0.058 & 0.066 & -0.06 & 0.187 & 2818 & 1\\
b\_public\_primary\_care & -0.021 & 0.067 & -0.152 & 0.1 & 3751 & 1\\
b\_inpatient\_public & -0.221 & 0.055 & -0.325 & -0.12 & 3352 & 1\\
b\_outpatient\_public & 0.178 & 0.073 & 0.037 & 0.31 & 3721 & 1\\
b\_healthcare\_regulator & -0.116 & 0.053 & -0.215 & -0.017 & 3834 & 1\\
b\_quality\_reports & 0.027 & 0.055 & -0.073 & 0.136 & 2532 & 1\\
b\_reports\_no\_choice & -0.321 & 0.084 & -0.478 & -0.162 & 2487 & 1\\
b\_single\_inpatient\_public & -0.346 & 0.077 & -0.483 & -0.197 & 3434 & 1\\
b\_single\_outpatient\_public & 0.089 & 0.072 & -0.046 & 0.225 & 3720 & 1\\
b\_single\_public\_primary\_care & -0.11 & 0.072 & -0.249 & 0.021 & 3830 & 1\\
\end{tabular}
\end{table}


Table \ref{tab:org73615e2} summarizes the posterior distributions for the extended model. The signs on the coefficients that were clearly signed before are similar here. The effects of inpatient/outpatient/primary care private/public are hard to compare due to the interaction with single payer which is not present in Table \ref{tab:org2052a02}.

We summarize the effects of single/multiple payers and private/public healthcare provision with Figure \ref{fig:org02f0695}. The estimated effects are relative to the outcome (normalized to 0) with multiple insurers and predominantly private providers. The line segments indicate the 95\% intervals of the effect. When healthcare provision is organized predominantly via private agents, we are reasonably sure (with a bit less than 95\% probability, as indicated by the line segment) that mortality is lower with multiple insurers than with a single payer. That is, most of the 95\% interval of the effect of a single payer lies above the effect for multiple insurers (normalized to 0) with private providers.

If in a country inpatient care is predominantly provided via public providers, (macro) mortality is substantially reduced with a single payer compared to multiple insurers. With outpatient care predominantly publicly provided, mortality tends to be lower with multiple insurers. But also with multiple insurers, mortality is lower with private providers. For primary care, there does not seem to be much of a difference between private providers with multiple insurers on the one hand and single/multiple payers with publicly provided primary care on the other hand.

\begin{figure}[htbp]
\centering
\includegraphics[width=.9\linewidth]{./figures/single_multiple_public_private.png}
\caption{\label{fig:org02f0695}Standardized mortality differentiated by number of insurers, private/public status of providers and different forms of care relative to the situation with multiple insurers and predominantly private provider.}
\end{figure}

Hence, an optimal system would feature private providers with multiple insurers for primary and outpatient care, while at  the same time a single payer with (predominantly) public provision of inpatient care.

The model in section \ref{sec:theory} suggests the following interpretation of these findings. Inpatient care requires high sunk costs to start a hospital where patients can stay the night. These costs are far higher than starting an outpatient or primary care clinic. Hence, inpatient hospital markets tend to be concentrated and these hospitals have high values for \(\Delta \nu\), the marginal value for the insurer to have it in its network. With a high value for \(\Delta \nu\), a single payer can reduce the rents going to the hospital compared to competing insurers. This beneficial is effect is especially big with (predominantly) public inpatient providers.


\section{Policy implications}
\label{sec:org807ef87}

In terms of policy variables, we can summarize our results as follows. We are mainly interested in the health system variables related to efficiency, quality and competition. But let's briefly look at the variables we use as controls. Not surprisingly, spending more on healthcare tends to reduce treatable mortality. For some expenditure categories this effect is stronger than for others. However, this does not signal a policy implication to improve the healthcare system because some expenditure categories are more closely associated with mortality reduction --but not system performance-- than others. Higher income per head and lower income inequality are associated with lower mortality but are not generally seen as health policy instruments. Women tend to have lower mortality than men and this gender gap falls with GDP per head. In terms of explicit lifestyle variables, our estimates confirm the well known fact that lower fractions of people smoking, consuming alcohol and having high BMI tend to reduce mortality in the population.

In terms of the policy variables that we are interested in, we start with efficiency focused measures. First, decisions on whether insurance should cover a (new) treatment based on cost effectiveness considerations is correlated with lower mortality. Since cost effectiveness is not directly correlated with mortality, we interpret this as an improvement to the healthcare system. Since many countries in our sample already use CE analysis, there is little to gain from this observation in Europe.

Second, over the range of values observed in the data, it seems a good idea to use nurses for a number of tasks that are currently done by doctors. For given expenditure levels, such an increase in the nurses/doctor ratio is associated with lower mortality. As shown in Figure \ref{fig:org225a9c0}, this ratio differs almost by a factor 5 between countries. Hence, many countries can improve the performance of their healthcare system without having to spend more money.

Finally, improving primary care provision which is reflected in a reduction of avoidable hospitalizations is also correlated with lower mortality. Giving more responsibility to general practitioners to coordinate primary care services can help people to stay out of hospital. Table \ref{tab:org97c867f} shows that the best performing country has less than 10\% of avoidable hospitalizations compared to the worst. Hence, also for this measure there is ample scope for improvement in our sample. 

The underlying idea of the efficiency variables is that they save money which can then be spent on other activities that help to reduce mortality and improve the performance of the system.

For the competition variables, we find the following effects. Creating provider competition by allowing patients to choose their hospital gives incentives to attract customers by increasing quality efforts which reduce mortality. This is in line with findings in \cite{NBERw19800} and \cite{teisberg}.

Publishing provider quality reports on top of allowing provider choice appears not to do much in terms of reducing mortality. In this sense, the policies are substitutes. If there is no free provider choice, it is important that the government publishes provider quality reports every year to discipline hospital managers. Based on posterior means, provider choice seems to have a bigger impact on mortality than publishing reports (without provider choice).

A single payer system reduces mortality in the case where inpatient care is predominantly done by public providers. Otherwise, a system with competing insurers seems to work well. One way to organize this could be to have basic insurance for inpatient care with public hospitals and a national single payer. Then for other treatments there can be separate insurance offered by competing insurers and (predominantly) private providers.

Finally, introducing a separate regulator focusing on the healthcare sector also improves the performance of the healthcare system.

\section{Robustness analysis}
\label{sec:orgc6dd814}

In Appendix \ref{app:robustness}, we do two robustness analyses with respect to the mortality variable that we use: preventable mortality and  life years lost, resp. The results are similar to what we find with the main specification above. Here we discuss the main differences.

The coefficient on HC6 (preventive care) is clearly negative in the model for preventable mortality, but the posterior for the coefficient on curative expenditure has some mass on positive values as well. Preventable mortality is affected by a number of other factors --compared to treatable mortality which is more closely linked to health system performance-- that the effect of cost effectiveness analysis on mortality is no longer clearly negative. The same is true for the effects of a single payer and public primary care.

Publishing provider quality reports now also reduces mortality in case of provider choice, but the effect is smaller than when reports are published without provider choice. Here the mean effect of free provider choice is bigger (in absolute value) than that of making reports public without provider choice.

With life years lost as dependent variable we have the following differences with the results of the main model above. High BMI does no longer obviously increase life years lost. Single payer and healthcare regulator do not clearly reduce years lost here, as they do in the baseline specification. Also here, the average effect of allowing for hospital choice is bigger (in absolute value) than the effect of making provider reports public.

An advantage of the life years lost measure is that it is relatively straightforward to put a value on the gains made by the different policy measures: multiply the value of a life year with life years gained (i.e. reduction in potential years of life lost). We illustrate this in Figure \ref{fig:org84df79c} where we increase total healthcare expenditure with \(\Delta HC\) and calculate whether life years gained multiplied by their value on the horizontal axis exceeds \(\Delta HC\). Since, we have a posterior distribution for \texttt{b\_hc}, there is a probability that the value of life years gained exceeds \(\Delta HC\). Appendix \ref{sec:PYLL} explains the details of this figure.

If we value an additional life year at 50k euro, there is a 95\% probability that the value gained exceeds the increase in healthcare expenditure in Latvia. For higher values of a life year, this probability is basically 1.

For the average European country the effect of increasing expenditure is not so clear cut. If we value a life year at 150k, there is a 40\% probability that the increase in expenditure is exceeded by the value of life years gained. At 200k this probability is approximately 80\%.

For the Netherlands an increase in expenditure does not cause such an increase in life years gained that this gain covers the extra expenditure. Does this imply that the Netherlands should reduce its healthcare expenditure? It clearly does not: there is more to healthcare than reducing mortality. To illustrate, expenditure on long term care may well be welfare enhancing while hardly affecting mortality.

Figure \ref{fig:org84df79c} does suggest that Latvia can spend more on healthcare (even while we ignore the marginal cost of public funds) but does not imply that the Netherlands spends too much on healthcare.


\begin{figure}[htbp]
\centering
\includegraphics[width=.9\linewidth]{./figures/value_life.png}
\caption{\label{fig:org84df79c}Probability that spending \texttt{value} extra on healthcare leads to at least one life year gained}
\end{figure}


\section{Discussion}
\label{sec:org7dec181}

Whereas many health papers nowadays work with individual level data, the starting point of this paper is: what can we learn from recently collected health data at the country level by Eurostat and the OECD? As the data is recent, there are missing values and not many observations. With Bayesian estimation we can avoid throwing away precious data and not impose extrapolated data points. Although pinpointing causal mechanisms is hard with macro data, we do believe that a number of relevant insights emerge from this analysis.

Our analysis confirms that income inequality and bad lifestyle choices increase mortality. More importantly, increasing the efficiency of the healthcare system by employing more nurses relative to doctors and improving the coordinating role of primary care improves the performance of the system. Competition between providers by allowing patients to choose their physicians leads to better outcomes. However, when it comes to insurance, it is not so clear that competition improves performance of the system. Our results suggest beneficial effects of a single payer system. Finally, we find some evidence that public provision of inpatient care leads to lower mortality than private provision.

The next step in this research is to examine whether the policy choices that we find to boost performance also helped to keep the system resilient in response to the corona crisis.

\bibliographystyle{plainnat}
\bibliography{../marginalinnovation}


\newpage
\appendix



\section{Appendix}
\label{sec:orga0811f5}

This appendix provides a description of the variables that we use, more details on the posterior distributions of the parameters in our models and on the estimation with preventable mortality and potential years of life lost.

Finally, we present an estimation with self-perceived health as dependent variable. This variable is even further removed from health system performance than life years lost. Although the results (naturally) become more noisy, a number of our findings above turn out to be rather robust.

\subsection{Data}
\label{sec:org1a1caa5}

Here we give more background on the variables that we use, the dimensions of each variable and more detail on the variables with missing values.

\subsubsection{Variables}
\label{sec:orgc48d0f3}

\begin{enumerate}
\item mortality
\label{sec:orgc7cad5f}
\label{sec:mortality}

\cite{OECD_avoidable_mortality} gives the following motivation for developing the treatable and preventable mortality measures: "Assessing the performance of health systems is of increasing importance in OECD and EU countries. While avoidable mortality indicators are not meant to be a definite measure of health system performance, they provide a starting point to assess the performance of public health and health care policies in avoiding premature mortality from preventable and treatable causes of death." This is the way in which we employ these measures in this paper, as a starting point to assess the performance of healthcare policies. 

In section 2 of \cite{OECD_avoidable_mortality}, preventable and treatable causes of mortality are defined as follows:
\begin{itemize}
\item "Preventable mortality: Causes of death that can be mainly avoided through effective public
\end{itemize}
health and primary prevention interventions (i.e. before the onset of diseases/injuries, to
reduce incidence)."
\begin{itemize}
\item "Treatable (or amenable) mortality: Causes of death that can be mainly avoided through
\end{itemize}
timely and effective health care interventions, including secondary prevention and treatment
(i.e. after the onset of diseases, to reduce case-fatality)"

Link to treatable and preventable mortality data: \url{https://ec.europa.eu/eurostat/data/database?node\_code=hlth\_cd\_apr}

For each of these mortality categories, an age threshold is used of 75 years. From the same publication: "It is recognised that the age threshold of 75 is arbitrary and only reflects a current definition of premature mortality". The idea is that they capture premature or untimely deaths which is not so clear cut for the elderly. Note that there is no claim that preventable or treatable mortality should (or can) be zero. Further, the mortality rates are age standardized where the weighting factor is the age distribution of the European standard population as defined by Eurostat in 2012.

We finish this discussions with some examples and their motivation from the OECD paper to get an idea of what is included under preventable and treatable mortality.

\begin{itemize}
\item HIV/AIDS is labeled as preventable (e.g. by using condoms) but not treatable;
\item tetanus neonatorum is preventable through vaccination;
\item alcohol-related deaths are preventable through public health interventions (e.g. alcohol control policies);
\item legionnaires disease is treatable through early detection and appropriate antibiotic treatment;
\item lymphoid leukaemia is treatable through early detection and appropriate treatment;
\item upper respiratory infections are treatable through appropriate treatment;
\item cervical cancer is 50:50 preventable (through vaccination and screening) and treatable;
\item transport accidents can be prevented through public policies like road safety measures;
\item intentional self-harm can be prevented through public (mental) health interventions.
\end{itemize}


The data link to potential years of life lost (PYLL) is: \url{https://ec.europa.eu/eurostat/databrowser/view/hlth\_cd\_apyll/default/table?lang=en}

According to \url{https://ec.europa.eu/eurostat/cache/metadata/en/hlth\_cdeath\_esms.htm}, "PYLL is an indicator estimating the potential years lost due to premature death, i.e. death before 70. It is calculated by summing the number of years between the age at death and 70 years for each premature death. PYLL rate is expressed per 100 000 age-standardised population under 70." That is, in contrast to treatable and preventable mortality here an age cut-off of 70 years is used.

\item expenditure
\label{sec:orgf0003ed}

The data for the expenditure variables can be found here: \url{https://ec.europa.eu/eurostat/data/database?node\_code=hlth\_sha11\_hc}

We use healthcare expenditures in euros per capita. As described on this page: \url{https://ec.europa.eu/eurostat/cache/metadata/en/hlth\_sha11\_esms.htm\#stat\_pres1612341832096}
"Healthcare functions relate to the type of need that current expenditure on healthcare aims to satisfy or the kind of objective pursued. We use the following items:

\begin{itemize}
\item "curative care, which means the healthcare services during which the principal intent is to relieve symptoms or to reduce the severity of an illness or injury, or to protect against its exacerbation or complication that could threaten life or normal function";
\item "rehabilitative care, which means the services to stabilise, improve or restore impaired body functions and structures, compensate for the absence or loss of body functions and structures, improve activities and participation and prevent impairments, medical complications and risks";
\item "long-term care (health), which means a range of medical and personal care services that are consumed with the primary goal of alleviating pain and suffering and reducing or managing the deterioration in health status in patients with a degree of long-term dependency";
\item "ancillary services (non-specified by function), which means the healthcare or long-term care related services non-specified by function and non-specified by mode of provision, which the patient consumes directly, in particular during an independent contact with the health system and that are not integral part of a care service package, such as laboratory or imaging services or patient transportation and emergency rescue";
\item "pharmaceuticals and other medical non-durable goods (non-specified by function), which means pharmaceutical products and non-durable medical goods intended for use in the diagnosis, cure, mitigation or treatment of disease, including prescribed medicines and over-the-counter drugs, where the function and mode of provision are not specified;
\item therapeutic appliances and other medical goods (non-specified by function), which means medical durable goods including orthotic devices that support or correct deformities and/or abnormalities of the human body, orthopaedic appliances, prostheses or artificial extensions that replace a missing body part, and other prosthetic devices including implants which replace or supplement the functionality of a missing biological structure and medico-technical devices, where the function and the mode of provision are not specified";
\item "preventive care, which means any measure that aims to avoid or reduce the number or the severity of injuries and diseases, their sequelae and complications; Preventive care includes interventions for both individual and collective consumption";
\item "governance, and health system and financing administration, which means services that focus on the health system rather than direct healthcare, direct and support health system functioning, and are considered to be collective, as they are not allocated to specific individuals but benefit all health system users".
\end{itemize}

\item lifestyle
\label{sec:orgf3448af}

Under lifestyle we categorize the female dummy as in most countries women tend to be/live healthier than men. We also include income variables since we know that higher income is associated with better health, \texttt{Log GDP per capita}. Based on GDP per head for 5 quintiles, we calculate \texttt{inequality} based on equation \eqref{eq:19}.

The link to data on GDP per capita:
\url{https://ec.europa.eu/eurostat/data/database?node\_code=nama\_10\_pc}

Data on income quantiles can be found here:
\url{https://ec.europa.eu/eurostat/databrowser/view/icw\_res\_02/default/table?lang=en}

Lifestyle variables:
\begin{itemize}
\item \texttt{smoking}: fraction of population in age range 15-64 who smoke more than 20 cigarettes per day (\url{https://ec.europa.eu/eurostat/databrowser/view/HLTH\_EHIS\_SK3I\_\_custom\_389732/default/table?lang=en});
\item \texttt{high bmi}: fraction of population (15-64 years) with body mass index (bmi) above 30(\url{https://ec.europa.eu/eurostat/databrowser/view/HLTH\_EHIS\_BM1I\_\_custom\_389481/default/table?lang=en});
\item \texttt{alcohol}: fraction of population (15-64) who every month (but not every week) experience "heavy drinking" --ingesting more than 60g of pure ethanol on a single occasion (\url{https://ec.europa.eu/eurostat/databrowser/view/hlth\_ehis\_al3e/default/table?lang=en}).
\end{itemize}

\item efficiency
\label{sec:orga6af16e}

\begin{itemize}
\item \texttt{nurses/doctors ratio} is based on number of practicing medical doctors per 100k (\url{https://ec.europa.eu/eurostat/databrowser/view/hlth\_rs\_prs1/default/table?lang=en}) and number of practicing nurses and midwives per 100k population (\url{https://ec.europa.eu/eurostat/databrowser/view/hlth\_rs\_prsns/default/table?lang=en});
\item \texttt{avoidable hospitalizations} is taken from \cite{oecdHospitalAdmissions}. We use asthma and COPD hospital admissions in adults (per 100k population). Since primary care can deal with most of these cases, this variable captures gate keeping and the level of organization at primary care level;
\item \texttt{ce analysis}: this dummy variable is taken from the OECD Health Systems Characteristics Survey (\url{https://qdd.oecd.org/data/HSC});  in particular, the variable equals 1 for countries that answer "yes" to the question "Do HTAs generally include results of economic evaluation? (Q62b)";
\end{itemize}

\item market
\label{sec:org8412a87}

The following variables are all taken from OECD Health Systems Characteristics Survey (\url{https://qdd.oecd.org/data/HSC}):
\begin{itemize}
\item \texttt{hospital choice}: equals 1 if the answer to "Are patients usually free to choose hospitals for in-patient care? (Q40a)" is either "Patients can choose any hospital without any consequence for the level of coverage" or "Patients are free to choose any hospital but they have financial incentives to choose some providers, please specify". Both answers imply that patients have (some) choice in choosing their hospital;
\item \texttt{single payer system}: equals 1 if the answer to the question "What is the main source of basic health care coverage in your country? (Q2)" is either "A national health system covering the country as a whole" or "A single health insurance fund (single-payer model)"; in either case, there is one payer/insurer bargaining with providers about treatment prices;
\item \texttt{public primary care}: equals 1 if "Publically employed" is the answer to the question "Are physicians supplying primary care services predominantly (Q29a)";
\item \texttt{inpatient public}: equals 1 if "Publically employed" is the answer to the question "Are physicians supplying in-patient specialist services predominantly (Q31a)";
\item \texttt{outpatient public}: equals 1 if the answer to "Are outpatient specialists' services provided predominantly in: (Q25a)" is either "Outpatient departments of public hospitals" or "Public multi-specialty clinic".
\end{itemize}


\item quality
\label{sec:org430a601}

\begin{itemize}
\item \texttt{healthcare regulator}: equals 1 if answer to "Is there an organisation with responsibility for national policy on health care quality in your country? (Q66)" is "Yes";
\item \texttt{public provider quality reports}: equals 1 if the answer to "Are these metrics publicly reported at the provider level at least annually? (Q71)" is "Yes", where these metrics refer to the question "Is there a set of national metrics available to monitor compliance with the standards in your country?".
\end{itemize}
\end{enumerate}



\subsubsection{Data dimensions}
\label{sec:orgfa39e19}
\label{sec:data_dimensions}

Table \ref{tab:orgb77b1e7} shows for each variable that we use the dimensions over which it varies. To illustrate, GDP per capita varies over countries and time but not gender. The income quintiles that we use for the inequality variable, we have for one year only. The lifestyle variable related to smoking varies with country and gender but we have it for one year only. All health system data from OECD is for one year only, except the avoidable hospitalizations variable which we have for the years 2012-2017.

The table further indicates whether the variable has missing values and the data source of the variable.

\begin{table}[htbp]
\caption{\label{tab:orgb77b1e7}Overview of dimensions per variable and the source.}
\centering
\begin{tabular}{lllllll}
category & variable & country & time & gender & missing & source\\
 &  &  &  &  & values & \\
\hline
mortality & treatable & x & x & x &  & Eurostat\\
 & preventable & x & x & x &  & Eurostat\\
 & PYLL & x & x & x &  & Eurostat\\
expenditure & HC (1-6) & x & x &  & x & Eurostat\\
lifestyle & GDP & x & x &  &  & Eurostat\\
 & inequality & x &  &  &  & Eurostat\\
 & smoking & x &  & x & x & Eurostat\\
 & high bmi & x &  & x & x & Eurostat\\
 & binge drinking & x &  & x & x & Eurostat\\
efficiency & nurses/doctor ratio & x &  &  & x & Eurostat\\
 & avoidable hospital. & x & x &  & x & OECD\\
 & CE analysis & x &  &  &  & OECD\\
market & hospital choice & x &  &  &  & OECD\\
 & single payer & x &  &  &  & OECD\\
 & public primary care & x &  &  &  & OECD\\
 & public inpatient care & x &  &  &  & OECD\\
 & public outpatient care & x &  &  &  & OECD\\
quality & healthcare regulator & x &  &  &  & OECD\\
 & provider reports public & x &  &  &  & OECD\\
\end{tabular}
\end{table}

Table \ref{tab:org98033cc} gives the number of values that are missing \textbf{in the final dataframe} for each variable that has missing values. To illustrate how to read this table, we consider two examples. As indicated in Table \ref{tab:orgd6eb02a}, we have data on 24 countries. For avoidable hospitalizations we have no data for the year 2011 which leads to \(24*2 = 48\) missing values (24 countries and 2 genders). Thus over the years 2012-2017 we also have missing values for some countries.

Since the lifestyle variables related to smoking and high bmi have no time dimension, 14 missing observations imply that we lack this information for 1 country (for 7 years and 2 genders); for both these variables we have no information for Switzerland.

\begin{table}[htbp]
\caption{\label{tab:org98033cc}Number of missing values per variable.}
\centering
\begin{tabular}{lr}
variable & number nan\\
\hline
avoidable hospitalizations & 108\\
HC2 & 84\\
HC1 & 70\\
binge drinking & 40\\
HC7 & 20\\
HC6 & 20\\
HC5 & 20\\
HC4 & 20\\
HC3 & 20\\
smoking & 14\\
high bmi & 14\\
\end{tabular}
\end{table}

\subsection{posterior plots}
\label{sec:org2c6fc8d}

This section presents the posterior plots for the parameters of the model in the main text. As explained in \cite{mcelreath}, the Markov Chain Monte Carlo algorithm that we use is guaranteed to converge in the long run to the posterior distribution; but it is not known how many samples is sufficient for this convergence. A trace plot helps to judge whether we have enough samples to trust the posterior distributions. The plot shows each sample drawn from the posterior connected by a line (figure on the right) and the posterior distributions for the four chains that we use (figure on the left).

We have more faith in our sampling process if the trace plot (right figure) satisfies three features. First, the plot should be stationary; for example, not trending upwards or downwards. In other words, the posterior mean value of the parameter is constant from beginning to end. Second, there should be good mixing which translates into the condensed zig-zagging of the trace. This implies that values are drawn across the whole domain of the posterior quickly after each other. In other words, the algorithm does not wander around for a while in one part of the posterior distribution and then "spends time" in another part of the distribution for a while. Finally, we want the different chains to cover the same regions. This is sometimes referred to as convergence. All three features are satisfied in the trace plots presented.


\subsubsection{expenditure}
\label{sec:org223eae6}
\label{app:expenditure}

Figure \ref{fig:org79091bd} shows the trace plots for the coefficients of the expenditure variables \texttt{b\_hc1} to \texttt{b\_hc7}. The coefficient on total expenditure \texttt{b\_hc} is not itself estimated but derived from the estimated coefficients as shown in equation \eqref{eq:18}. As the variables in our analysis are standardized, \texttt{b\_hc} actually equals the sum of \texttt{b\_hci} where each coefficient is divided by the standard deviation of \(\ln(HC_i)\) for \(i=1,2,...,7\).

\begin{figure}[htbp]
\centering
\includegraphics[width=.9\linewidth]{./figures/trace_expenditure.png}
\caption{\label{fig:org79091bd}Trace plot for parameters related to expenditure}
\end{figure}


\subsubsection{health}
\label{sec:org21bab84}

Figure \ref{fig:orgb77adc4} shows the trace plots for the variables related to health/lifestyle.

\begin{figure}[htbp]
\centering
\includegraphics[width=.9\linewidth]{./figures/trace_health.png}
\caption{\label{fig:orgb77adc4}Trace plot for parameters related to health}
\end{figure}


\subsubsection{efficiency}
\label{sec:orga74644e}

The traces for the variables related to efficiency are summarized in Figure \ref{fig:orgc7173ee}.

\begin{figure}[htbp]
\centering
\includegraphics[width=.9\linewidth]{./figures/trace_efficiency.png}
\caption{\label{fig:orgc7173ee}Trace plot for parameters related to efficiency}
\end{figure}

\subsubsection{market}
\label{sec:org3fa64e1}

The trace plots for the variables related to the market are presented in Figure \ref{fig:org7b61654}.

\begin{figure}[htbp]
\centering
\includegraphics[width=.9\linewidth]{./figures/trace_market.png}
\caption{\label{fig:org7b61654}Trace plot for parameters related to market}
\end{figure}


\subsubsection{quality}
\label{sec:org96e18dc}

The trace plots for the healthcare regulator and yearly publishing of provider quality reports are presented in Figure \ref{fig:org6c251d4}.

\begin{figure}[htbp]
\centering
\includegraphics[width=.9\linewidth]{./figures/trace_quality.png}
\caption{\label{fig:org6c251d4}Trace plot for parameters related to quality}
\end{figure}
\subsubsection{interaction effects}
\label{sec:org7bf23d5}

Figure \ref{fig:org3278d6e} gives the trace plot for the coefficient on the interaction between publishing quality reports in case there is no free provider choice.

\begin{figure}[htbp]
\centering
\includegraphics[width=.9\linewidth]{./figures/trace_interaction.png}
\caption{\label{fig:org3278d6e}Trace plot for parameters related to interaction effects}
\end{figure}



\subsection{Robustness checks}
\label{sec:orgbe57f6a}
\label{app:robustness}

Here we look at two robustness checks for the dependent variable that we use in this paper. Hence, we run the model in the main text for preventable mortality and potential years of life lost instead of treatable mortality. The results are similar to the ones found above.

In the on-line appendix we also consider self-perceived health as dependent variable. This is affected by a lot of other variables not directly related to health system performance. Hence, the estimated effects are more noisy than for the avoidable mortality measures that we use. However, a number  of the effects that we found above are robust to using this measure as dependent variable.

\subsubsection{preventable mortality}
\label{sec:orgc515c0c}

In this section we run the model with preventable mortality (instead of treatable mortality) as a robustness check. The posterior of the coefficients are similar to the what we find in the main text. Table \ref{tab:org2e65b0a} summarizes the posterior for the coefficients of the model.

There are three differences with the posteriors found in the main text. With preventable mortality cost effectiveness analysis and public primary care no longer have clear mortality reducing effects while quality reports now also reduce mortality in case patients choose their own hospital. However, the latter effect is smaller than the effect of public reports when patients cannot freely choose their provider. So also here, free provider choice and public reports are partial substitutes.

\begin{table}[htbp]
\caption{\label{tab:org2e65b0a}Summary posterior distributions of the model's parameters with preventable mortality}
\centering
\begin{tabular}{lrrrrrr}
coefficient & mean & sd & hpd\_3\% & hpd\_97\% & ess\_mean & r\_hat\\
\hline
b\_hc & -0.215 & 0.086 & -0.372 & -0.048 & 3164 & 1\\
b\_hc1 & -0.014 & 0.026 & -0.063 & 0.036 & 1316 & 1\\
b\_hc2 & 0 & 0.026 & -0.048 & 0.049 & 802 & 1\\
b\_hc3 & 0.129 & 0.048 & 0.039 & 0.218 & 2434 & 1\\
b\_hc4 & -0.035 & 0.026 & -0.084 & 0.015 & 2636 & 1\\
b\_hc5 & -0.059 & 0.035 & -0.126 & 0.006 & 3484 & 1\\
b\_hc6 & -0.108 & 0.032 & -0.17 & -0.048 & 2902 & 1\\
b\_hc7 & 0.018 & 0.027 & -0.032 & 0.069 & 3651 & 1\\
b\_female & -0.995 & 0.041 & -1.067 & -0.913 & 3316 & 1\\
b\_gdp & -0.162 & 0.053 & -0.261 & -0.061 & 3741 & 1\\
b\_female\_gdp & 0.094 & 0.025 & 0.046 & 0.139 & 4565 & 1\\
b\_inequality & 0.172 & 0.02 & 0.134 & 0.21 & 1894 & 1\\
b\_smoking & 0.206 & 0.026 & 0.158 & 0.257 & 3285 & 1\\
b\_obese & 0.133 & 0.016 & 0.103 & 0.162 & 3873 & 1\\
b\_alcohol & 0.258 & 0.024 & 0.214 & 0.303 & 2309 & 1\\
b\_nurses\_doctors & -0.192 & 0.015 & -0.222 & -0.164 & 1920 & 1\\
b\_avoidable\_hospitalizations & 0.089 & 0.014 & 0.063 & 0.117 & 1939 & 1\\
b\_ce\_analysis & -0 & 0.07 & -0.126 & 0.134 & 3432 & 1\\
b\_hospital\_choice & -0.448 & 0.049 & -0.536 & -0.356 & 2316 & 1\\
b\_single & 0.04 & 0.049 & -0.052 & 0.131 & 2556 & 1\\
b\_public\_primary\_care & 0.06 & 0.037 & -0.008 & 0.131 & 3534 & 1\\
b\_outpatient\_public & 0.201 & 0.05 & 0.107 & 0.291 & 2247 & 1\\
b\_inpatient\_public & -0.344 & 0.046 & -0.428 & -0.256 & 2420 & 1\\
b\_healthcare\_regulator & -0.157 & 0.044 & -0.236 & -0.068 & 4051 & 1\\
b\_quality\_reports & -0.169 & 0.046 & -0.254 & -0.082 & 1795 & 1\\
b\_reports\_no\_choice & -0.218 & 0.089 & -0.392 & -0.056 & 1674 & 1\\
\end{tabular}
\end{table}



\subsubsection{Potential Years of Life Lost (PYLL)}
\label{sec:orgce31dca}
\label{sec:PYLL}

The results in Table \ref{tab:orge171c7e} are comparable to the model in the main text. The exceptions are that a high bmi no longer clearly increases life years lost and a healthcare regulator does no longer reduce life years lost with high probability. In other words, there is some posterior probability that a high bmi reduces PYLL while having a healthcare regulator increases PYLL.

Although the variable PYLL is broader and hence less clearly associated with the quality of the healthcare system than treatable and preventable mortality, it has the advantage that it can be more easily linked to a euro value. If we are willing to put a value, say \(v\), on a life year, a policy change that reduces PYLL by one year increases value by \(v\). 

Suppose increasing healthcare expenditure by \(y\) euro per year, reduces PYLL by \(x\) years (per year), then this investment is worth it based on mortality only (i.e. ignoring quality of life aspects) if \(v x - y \geq 0\). This is the exercise we do below by considering a range of values for \(v\).

\begin{table}[htbp]
\caption{\label{tab:orge171c7e}Summary posterior distributions of the model's parameters with life years lost}
\centering
\begin{tabular}{lrrrrrr}
coefficient & mean & sd & hpd\_3\% & hpd\_97\% & ess\_mean & r\_hat\\
\hline
b\_hc & -0.43 & 0.098 & -0.615 & -0.247 & 2623 & 1\\
b\_hc1 & -0.074 & 0.025 & -0.124 & -0.028 & 1446 & 1\\
b\_hc2 & 0.053 & 0.025 & 0.008 & 0.1 & 990 & 1.01\\
b\_hc3 & 0.069 & 0.046 & -0.015 & 0.156 & 3540 & 1\\
b\_hc4 & -0.071 & 0.032 & -0.132 & -0.014 & 3139 & 1\\
b\_hc5 & -0.082 & 0.038 & -0.152 & -0.008 & 3122 & 1\\
b\_hc6 & -0.118 & 0.035 & -0.181 & -0.051 & 3911 & 1\\
b\_hc7 & -0 & 0.031 & -0.058 & 0.057 & 3099 & 1\\
b\_female & -0.897 & 0.044 & -0.978 & -0.813 & 3634 & 1\\
b\_gdp & -0.303 & 0.059 & -0.412 & -0.192 & 2979 & 1\\
b\_female\_gdp & 0.146 & 0.027 & 0.095 & 0.195 & 4845 & 1\\
b\_inequality & 0.242 & 0.024 & 0.197 & 0.288 & 2167 & 1\\
b\_smoking & 0.165 & 0.028 & 0.113 & 0.217 & 3217 & 1\\
b\_obese & 0.022 & 0.018 & -0.011 & 0.057 & 3861 & 1\\
b\_alcohol & 0.312 & 0.029 & 0.256 & 0.367 & 2253 & 1\\
b\_nurses\_doctors & -0.103 & 0.017 & -0.135 & -0.071 & 1546 & 1\\
b\_avoidable\_hospitalizations & 0.071 & 0.016 & 0.044 & 0.102 & 2805 & 1\\
b\_ce\_analysis & -0.161 & 0.072 & -0.291 & -0.025 & 4991 & 1\\
b\_hospital\_choice & -0.531 & 0.051 & -0.629 & -0.437 & 3536 & 1\\
b\_single & -0.045 & 0.056 & -0.148 & 0.058 & 2663 & 1\\
b\_public\_primary\_care & -0.155 & 0.048 & -0.245 & -0.065 & 2495 & 1\\
b\_outpatient\_public & 0.171 & 0.052 & 0.075 & 0.271 & 3555 & 1\\
b\_inpatient\_public & -0.3 & 0.049 & -0.39 & -0.207 & 2571 & 1\\
b\_healthcare\_regulator & -0.046 & 0.049 & -0.138 & 0.044 & 4355 & 1\\
b\_quality\_reports & -0.068 & 0.051 & -0.16 & 0.033 & 2474 & 1\\
b\_reports\_no\_choice & -0.294 & 0.083 & -0.45 & -0.141 & 3207 & 1\\
\end{tabular}
\end{table}

We need the following steps to derive Figure \ref{fig:org84df79c}. First, we estimate our model using standardized variables. Let \(L\) denote PYLL, then the part of the estimated equation that we are interested in here can be written as:
\begin{equation}
\label{eq:27}
\frac{\ln(L)-E(\ln(L))}{SD(\ln(L))} = b_{hc} \ln(HC) + ...
\end{equation}
As explained in Section \ref{app:expenditure}, \(b_{hc}\) is already corrected for the standard deviations of the \(hc_i\) terms. Taking derivatives we find
\begin{equation}
\label{eq:31}
\frac{d \ln(L)}{d \ln(HC)} = b_{hc} SD(\ln(L))
\end{equation}
and thus
\begin{equation}
\label{eq:32}
\frac{dL}{dHC} = \frac{L}{HC} b_{hc} SD(\ln(L))
\end{equation}
The derivative \(-dL/dHC\) gives the gain in life years (reduction in life years lost) per 100,000 age-standardized population under 70 of spending 1 euro more per capita on healthcare. The cost of this 1 euro per head, translates into 100,000 euro for this group. Hence, we want to know whether \(100,000 \leq v (-dL/dHC)\), where \(v\) denotes the value of a life year gained. 

Using equation \eqref{eq:32}, we write
\begin{equation}
\label{eq:22}
v \frac{L}{HC} (-b_{hc}) SD(\ln(L)) \geq 100,000
\end{equation}
where we use the posterior distribution for \texttt{b\_hc}. Thus we calculate the posterior probability that spending an extra euro per head yields more value than this costs for different values of \(v\). Figure \ref{fig:org84df79c} shows these probabilities for Latvia, the Netherlands and an average European country (taking the average across countries unweighted by population size; hence the figure does not apply to the average European citizen).

In Latvia, choosing \(v = 100,000\) or higher leads to an increase in value that exceeds the cost of extra expenditure with (almost) probability 1.0. Since 100,000 euro for a life year is not an uncommon valuation, this indicates that Latvia may want to consider an increase in healthcare expenditure. However, note that our derivation above does not take the marginal cost of public funds into account. If the taxation needed to finance this increase in healthcare expenditure is "very distortionary", such an increase in expenditure may actually not be welfare enhancing.

At the other extreme is the Netherlands where even with \(v = 300,000\), it is not worthwhile to increase healthcare expenditures. Does this imply that Dutch healthcare expenditures should be reduced? The answer is no for at least two reasons. First, healthcare is as much about quality of life as mortality. Hence, expenditure that increases quality of life but does not reduce mortality is valuable but not picked up by the calculation underlying the figure. Second, PYLL refers to life years gained till the age of 70. Hence, life years gained beyond 70 (e.g. mortality reduced for a 72 year old) are not valued in our calculation.

In this sense, Figure \ref{fig:org84df79c} underestimates the value of increasing healthcare expenditures. As mentioned, we underestimate the cost of healthcare expenditures in so far as we do not take the marginal cost of public funds into account.
\end{document}